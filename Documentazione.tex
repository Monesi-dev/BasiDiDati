\documentclass{article}

% Language setting
% Replace `english' with e.g. `spanish' to change the document language
\usepackage[english]{babel}
\usepackage{soul}
\usepackage{color}
\usepackage{titlesec}                   % Used to increase spacing after title

% Set page size and margins
% Replace `letterpaper' with `a4paper' for UK/EU standard size
\usepackage[letterpaper,top=4cm,bottom=6cm,left=5cm,right=5cm,marginparwidth=2cm]{geometry}

% Bibliography
\usepackage{biblatex}       %   Imports biblatex package
\addbibresource{sample.bib} %   Imports the bibliography file
\usepackage{csquotes}

% Useful packages
\usepackage{amsmath}
\usepackage{listings} % To embed snippets of code
\usepackage{graphicx}
\usepackage[colorlinks=true, allcolors=blue]{hyperref}

% Used to enable \floor
\usepackage{mathtools}
\DeclarePairedDelimiter\ceil{\lceil}{\rceil}
\DeclarePairedDelimiter\floor{\lfloor}{\rfloor}

% Used to Center Text in Tables
\usepackage{array}                      
\newcolumntype{P}[1]{>{\centering\arraybackslash}p{#1}}
\newcolumntype{C}[1]{>{\centering\arraybackslash}m{#1}}


\definecolor{dkgreen}{rgb}{0,0.6,0}
\definecolor{gray}{rgb}{0.5,0.5,0.5}
\definecolor{mauve}{rgb}{0.58,0,0.82}

\lstset{frame=tb,
  language=SQL,
  aboveskip=3mm,
  belowskip=3mm,
  showstringspaces=false,
  columns=flexible,
  basicstyle={\small\ttfamily},
  numbers=none,
  numberstyle=\tiny\color{gray},
  keywordstyle=\color{blue},
  commentstyle=\color{dkgreen},
  stringstyle=\color{mauve},
  breaklines=true,
  breakatwhitespace=true,
  tabsize=3
}

\title{Progetto Basi di Dati}
\author{Simone Gallo, Riccardo Ciucci}   
\date{ }                                % This hides the Date

\begin{document}
\titlespacing*{\subsection}             % Modifying Subsection Default Spacing
  {0pt}{2\baselineskip}{2\baselineskip}
\titlespacing*{\section}                % Modifying Section Default Spacing
  {0pt}{2\baselineskip}{2\baselineskip}
\begin{figure}
    \centering
    \includegraphics[width=0.5\linewidth]{assets/Logo UniPI.png}
    \label{fig:Logo Università di Pisa}
\end{figure}

\maketitle

\newpage

% Index
{ 
\renewcommand{\contentsname}{Indice}    % This changes the default name of \tableofcontents (Contents -> Indice)
\hypersetup{hidelinks}                  % This changes the color of \tableofcontents from blue to black
\tableofcontents                        % This generates the "index" even though it is not exactly an index    
}       

\newpage                                % This is used to start a new page 
%
%
%
% Fine Pagina 3 - Indice
%
% ~ ~ ~ ~ ~ ~ ~ ~ ~ ~ ~ ~ ~ ~ ~ 
%
% Inizio Pagina 4  -  Introduzione
%
%
%
\section{Introduzione} 
Si desidera progettare e realizzare una base di dati che permetta sia la memorizzazione di dati ed informazioni relativi a FilmSphere, un servizio internazionale di streaming di contenuti video online, che l'utilizzo di operazioni e funzioni di data analytics che permettono di migliorare l'esperienza dell'utente. \\ \\ \\
Il gestore del sistema consente l'inserimento di Film e le informazioni ad essi correlate, la memorizzazione dei vari formati utilizzati insieme alle relative versioni di questi ultimi ed, infine, l'immagazzinamento di dati che riguardano gli utenti compresi abbonamento, fatture, con eventuali pagamenti effettuati, e raccomandazione di contenuti attraverso algoritmi di valutazione di contenuti.  \\ \\
Oltretutto il suddetto sistema sovraintende e gestisce anche la CDN, ossia la rete di distribuzione del servizio costituita da server dislocati nel globo, grazie all'implementazione di funzionalita' lato server di analisi di dati, come il caching. \\ \\ \\
Ad ogni componente dello schema ER che riguarda una delle diverse Aree della base di dati viene univocamente associato un colore: \\
\begin{itemize}
  \item \sethlcolor{red}\hl{Area Contenuti}
  \item \sethlcolor{green}\hl{Area Formati}
  \item \sethlcolor{cyan}\hl{Area Clienti} 
  \item \sethlcolor{yellow}\hl{Area Streaming}
\end{itemize}
%
%
%
% Fine Pagina 4
%
% ~ ~ ~ ~ ~ ~ ~ ~ ~ ~ ~ 
%
% Inizio Pagina 5
%
%
%
\section{Progettazione Concettuale}
\subsection{Dizionario dell'Entità}
\begin{tabular}{|C{1.9cm}|C{3.2cm}|C{2.3cm}|C{4cm}|}
\hline
 \textbf{Entità}& \textbf{Attributi}& \textbf{Identificatore}& \textbf{Descrizione}\\ 
\hline
\hline
& & & \\
 \sethlcolor{red}\hl{Film*}   & Id,  Descrizione, Titolo & ID &   Film offerto dal servizio di streaming online \\
& & & \\    
\hline
& & & \\
 \sethlcolor{red}\hl{Genere} & Nome & Nome & Genere dei Film offerti da FilmSpere \\
 & & & \\
\hline
& & & \\
 \sethlcolor{red}\hl{Attore*} & Nome, Cognome, Popolarità & Nome, Cognome & Specializzazione di Artista, attore recitante in un Film \\
& & & \\
\hline
& & & \\
 \sethlcolor{red}\hl{Regista*} & Nome, Cognome, Popolarità & Nome, Cognome & Specializzazione di Artista, regista dirigente un Film \\
& & & \\
\hline
& & & \\
 \sethlcolor{red}\hl{Artista*} & Nome, Cognome, Popolarità & Nome, Cognome  & Generalizzazione di Attore e Regista \\
& & & \\
\hline
& & & \\
 \sethlcolor{red}\hl{Premio*} & Macrotipo, Microtipo & Macrotipo, Microtipo  & Premio attribuito a Film, Attori e/o Registi \\
& & & \\
\hline
& & & \\
 \sethlcolor{red}\hl{Vincita*} & Data & Data, Macrotipo(FK), Microtipo(FK) & Vincita di un Premio da parte di Film, Attori e/o Registi \\
& & & \\
\hline
\end{tabular}
\newpage
%
%
%
% Fine Pagina 5
%
% ~ ~ ~ ~ ~ ~ ~ ~ ~ ~ ~ 
%
% Inizio Pagina 6
%
%
%
\begin{tabular}{|C{1.9cm}|C{3.2cm}|C{2.3cm}|C{4cm}|}
\hline
 \textbf{Entità}& \textbf{Attributi}& \textbf{Identificatore}& \textbf{Descrizione}\\ 
\hline
\hline
& & & \\
 \sethlcolor{red}\hl{Recensore*} & Codice, Nome, Cognome & Codice & Generalizzazione di Utente e Critico \\
& & & \\
\hline
& & & \\
 \sethlcolor{red}\hl{Critico*} & Codice, Nome, Cognome & Codice & Critico Cinematografico \\
& & & \\
\hline
& & & \\
 \sethlcolor{red}\hl{Casa- Produzione} & Nome & Nome & Casa di Produzione Cinematografica \\
& & & \\
\hline
& & & \\
 \sethlcolor{red}\hl{Paese*} & Codice, Nome, Posizione & Codice & Paese dal quale provengono dei Film \\
& & & \\
\hline
& & & \\
 \sethlcolor{green}\hl{File*} & Id, Dimensione, LingueDoppiaggio, LingueSottotitoli, BitRate, FormatoContenitore & Id & File in cui è stato salvato un Formato di un Film \\ 
& & & \\
\hline
& & & \\
 \sethlcolor{green}\hl{Edizione} & Id, Anno, Rapporto d'Aspetto, Lunghezza & Id & Edizione di un Film (e.g. "Godfather" fu rilasciato una seconda volta nel 1977 con alcune differenze) \\ 
& & & \\
\hline 
& & & \\
 \sethlcolor{green}\hl{Formato- Codifica*} & Famiglia, Versione, MaxBitrate, Lossy & Famiglia, Versione & Generalizzazione di Audio e Video, Formato di Codifica e Decodifica \\
& & & \\
\hline 
& & & \\
 \sethlcolor{green}\hl{Audio*} & Famiglia, Versione, MaxBitrate, Lossy & Famiglia, Versione & Specializzazione di Formato Codifica\\ 
& & & \\
\hline 
\end{tabular}
%
%
%
% Fine Pagina 6
%
% ~ ~ ~ ~ ~ ~ ~ ~ ~ ~ ~
%
% Inizio Pagina 7
%
%
%
\begin{tabular}{|C{1.9cm}|C{3.2cm}|C{2.3cm}|C{4cm}|}
\hline
 \textbf{Entità}& \textbf{Attributi}& \textbf{Identificatore}& \textbf{Descrizione}\\ 
\hline
\hline
& & & \\
 \sethlcolor{green}\hl{Video*} & Famiglia, Versione, Anno, Bitrate, Lossy & Famiglia, Versione & Specializzazione di Formato Codifica \\
& & & \\ 
\hline
& & & \\
 \sethlcolor{yellow}\hl{Server} & Id, Lunghezza Banda, MTU, MaxConessioni, Carico Attuale, Posizione & Id & Server appartente alla rete distribuita \\
& & & \\
\hline 
& & & \\
 \sethlcolor{yellow}\hl{IP Range} & Inizio, Fine, DataInizio, DataFine & Inizio, Fine, DataInizio & Range di IP appartenenti ad una Zona Geografia \\
& & & \\
\hline 
& & & \\
 \sethlcolor{cyan}\hl{Utente} & Codice, Nome, Cognome, Email, Password & Codice & Utente registrato al servizio di streaming \\ 
& & & \\
\hline 
& & & \\
 \sethlcolor{cyan}\hl{Connessione} & IP, Inizio, Fine, Hardware & IP, Inizio, Codice(FK) & Connessione al servizio da parte di un Utente\\ 
& & & \\ 
\hline
& & & \\
 \sethlcolor{cyan}\hl{Fattura} & ID & ID & Fattura emessa da FilmSphere \\ 
& & & \\
\hline
& & & \\
 \sethlcolor{cyan}\hl{Pagamento*} & Data, Carta di Credito & ID(FK) & Pagamento effettuato da un Utente per una Fattura \\
& & & \\
\hline
\end{tabular}
%
%
%
% Fine Pagina 7
%
% ~ ~ ~ ~ ~ ~ ~ ~ ~ ~ ~ 
%
% Inizio Pagina 8
%
%
%
\begin{tabular}{|C{1.9cm}|C{3.2cm}|C{2.3cm}|C{4cm}|}
\hline
\textbf{Entità}& \textbf{Attributi}& \textbf{Identificatore}& \textbf{Descrizione}\\ 
\hline
\hline
& & & \\
 \sethlcolor{cyan}\hl{Visualizzazione} & Timestamp & Timestamp, ID(FK), Codice(FK) & Visualizzazione di un File da parte di un Utente \\ 
& & & \\
\hline 
& & & \\
 \sethlcolor{cyan}\hl{Abbonamento} & Tipo, Tariffa, Durata, Definizione, Offline, GB Mensili, Max Ore & Tipo & Abbonamento fornito dal servizio di streaming\\
& & & \\
\hline
\end{tabular}
\\ \\  
\\ \\
    Le entità indicate con "\textbf{*}" sono generalizzazione, specializzazioni o altre entità che sono state modificate durante la ristrutturazione.
\\ 
\\
In particolare durante la ristrutturazione sono state eliminate le seguenti entità: \textbf{Attore}, \textbf{Regista}, \textbf{Recensore}, \textbf{Audio}, \textbf{Video}. 
\\
Sono state modificate le seguenti: \textbf{File}, \textbf{Pagamento}, \textbf{Film}.
\\
Mentre sono state aggiunte le seguenti: \textbf{Lingua}, \textbf{Carta di Credito}.
\\
\\
\begin{tabular}{|C{1.3cm}|C{2.8cm}|C{2.3cm}|C{4cm}|}
\hline
\textbf{Entità}& \textbf{Attributi}& \textbf{Identificatore}& \textbf{Descrizione}\\ 
\hline
\hline
 \sethlcolor{red}\hl{Film} & ID, Titolo, Descrizione, MediaRecensioni, NumeroRecensioni & ID & Film offerto dal servizio di streaming online \\
\hline
 \sethlcolor{green}\hl{File} & ID, Dimensione, BitRate, FormatoContenitore & ID & File in cui è stato salvato un Film\\
\hline
 \sethlcolor{cyan}\hl{Pagamento} & Data & ID(FK) & Pagamento effettuato da un Utente per una Fattura\\
\hline
 \sethlcolor{red}\hl{Lingua} & Nome & Nome & Lingua in cui possono essere disponibili sia audio che sottotitoli\\
\hline
 \sethlcolor{cyan}\hl{Carta di Credito} & PAN, Data di Scadenza, CVV & PAN & Carta di Credito utilizzata per un Pagamento\\
\hline
\end{tabular}
%
%
%
% Fine Pagina 8
%
% ~ ~ ~ ~ ~ ~ ~ ~ ~ 
%
% Inizio Pagina 9
% 
%
%
\subsection{Dizionario delle Relazioni}
\begin{tabular}{|C{2cm}|C{4.5cm}|C{2.4cm}|C{2cm}|}
\hline
 \textbf{Relazione}& \textbf{Descrizione}& \textbf{Componenti}& \textbf{Attributi}\\ 
\hline
\hline
& & & \\
 \sethlcolor{red}\hl{Vincita- Artista} & Vincita, da parte di un Artista, di un premio & Vincita (0, 1), Artista (0, N) & -  \\ 
& & & \\
\hline
& & & \\
 \sethlcolor{red}\hl{Vincita- Premio} & Vincita di un particolare Premio & Vincita (1, 1), Premio (1, N) & - \\ 
& & & \\
\hline
& & & \\
 \sethlcolor{red}\hl{Vincita- Film} & Vincita, da parte di un Film, di un premio &  Vincita (1, 1), Film (0, N) & - \\
& & & \\
\hline
& & & \\
 \sethlcolor{red}\hl{Recitazione*} & Recitazione da parte di un Attore in un Film & Film (1, N), Attore (1, N) & - \\ 
& & & \\
\hline
& & & \\
 \sethlcolor{red}\hl{Regia*} & Direzione da parte di un Regista di un Film & Regista (1, N), Film (1, 1) & - \\
& & & \\
\hline 
& & & \\
 \sethlcolor{red}\hl{Genere- Film} &  Appartenenza di un Film ad uno o più Generi & Genere (1, N), Film (1, N) & - \\
& & & \\
\hline 
& & & \\
 \sethlcolor{red}\hl{Critica} & Valutazione, da parte di un Critico, di un Film & Critico (1, N), Film (0, N) & Testo, Data, Voto \\ 
& & & \\
\hline
& & & \\
 \sethlcolor{red}\hl{Recensione} & Valutazione, da parte di un Utente, di un Film & Utente (0, N), Film (0, N) & Voto \\
& & & \\
\hline 
& & & \\
 \sethlcolor{red}\hl{Produzione} & Produzione, da parte di una Casa di Produzione, di un Film & CasaPro- duzione (1, N), \,\,\,\,\,\,\,\,\,\,\,\, Film (1, 1) & Anno \\
& & & \\
\hline 
\end{tabular}
%
%
%
% Fine Pagina 9
%
% ~ ~ ~ ~ ~ ~ ~ ~ ~ ~ ~ 
% 
% Inizio Pagina 10 
%
%
%
\begin{tabular}{|C{2cm}|C{4.5cm}|C{2.4cm}|C{2cm}|}
\hline
 \textbf{Relazione}& \textbf{Descrizione}& \textbf{Componenti}& \textbf{Attributi}\\ 
\hline
\hline
& & & \\
 \sethlcolor{red}\hl{Appartenenza} & Residenza di una Casa di Produzione in un Paese &  CasaPro- duzione (1, 1), \,\,\,\,\,\,\, Paese (0, N)  & - \\ 
 & & & \\
\hline 
& & & \\
 \sethlcolor{green}\hl{Restrizione} & Impossibilità di vedere un Edizione in un Paese & Edizio- \,\,\,\,\,\,\,\,\,\,\,\,\,\,\,ne (0, N), Paese (0, N) & - \\
& & & \\
\hline 
& & & \\
 \sethlcolor{green}\hl{Edizione- File} &  Corrispondenza di uno o piu' File ad un'Edizione di un film & File (1, 1), Edizione (1, N)  & - \\
& & & \\
\hline 
& & & \\
 \sethlcolor{green}\hl{VideoFile*} & Presenza di un Formato Video in un File & Video (1, N), File (1, 1) & Qualità \\ 
& & & \\
\hline 
& & & \\
 \sethlcolor{green}\hl{AudioFile*} & Presenza di un Formato Audio in un File & Audio (1, N), File (1, 1) & Qualità \\ 
& & & \\
\hline
& & & \\
 \sethlcolor{yellow}\hl{P.o.P.} &  Presenza di un File in un Server & File (1, N), Server (1, N) & - \\
 & & & \\
\hline 
& & & \\
 \sethlcolor{yellow}\hl{Geolocal} & Corrispondenza di un IP Range ad un Paese & Paese (1, N), IP Range (1, 1) & - \\ 
& & & \\
\hline 
& & & \\
 \sethlcolor{yellow}\hl{Erogazione} & Erogazione, da parte di un Server, di un file per la Visualizzazione di un utente & Paese (1, N), IP Range (1, 1) & Inizio- Erogazione \\ 
& & & \\
\hline 
\end{tabular}
%
%
%
% Fine Pagina 10
%
% ~ ~ ~ ~ ~ ~ ~ ~ ~ ~ ~
%
% Inizio Pagina 11
%
%
%
\begin{tabular}{|C{2cm}|C{4.5cm}|C{2.4cm}|C{2cm}|}
\hline
 \textbf{Relazione}& \textbf{Descrizione}& \textbf{Componenti}& \textbf{Attributi}\\ 
\hline
\hline
& & & \\
 \sethlcolor{cyan}\hl{Connes- sioneUtente} & Corrispondenza tra un Utente e la sua Connessione & Utente (0, N), Conessione (1, 1) & - \\ 
& & & \\
\hline 
& & & \\
 \sethlcolor{cyan}\hl{Emissione} & Emissione di una Fattura per un Utente & Utente (0, N), Fattura (1, 1) & Data \\ 
& & & \\
\hline
& & & \\
 \sethlcolor{cyan}\hl{Pagamen- toFattura} & Corrispondenza tra una Fattura ed il suo, eventuale, Pagamento & Fattura (0, 1), Pagamento (1, 1) & - \\ 
& & & \\
\hline
& & & \\
 \sethlcolor{cyan}\hl{VisualizzazioneConnessione} &  Corrispondenza tra la Visualizzazione e la Connessione di un utente & Connessione (0, N), Visualizzazione (1, 1) & - \\ 
& & & \\
\hline 
& & & \\
 \sethlcolor{cyan}\hl{Abbona- mento Utente} &  Corrispondenza tra un Utente ed il suo ultimo Abbonamento & Utente (1, 1), Abbonamento (0, N) & DataInizio \\
& & & \\
\hline 
& & & \\
 \sethlcolor{cyan}\hl{Visualiz- zazioneEdizione} & Corrispondenza tra la Visualizzazione e l'Edizione visionata & Visualizzazione (1, 1), Edizione (0, N) & - \\ 
& & & \\
\hline
& & & \\
 \sethlcolor{cyan}\hl{Esclusione} & Esclusione di un Genere per un Abbonamento & Abbonamento (0, N), Genere (0, N) & - \\ 
& & & \\
\hline
\end{tabular}
\\ \\ 
%
%
%
% Fine Pagina 11
%
% ~ ~ ~ ~ ~ ~ ~ ~ ~ ~ 
%
% Inizio Pagina 12 
%
%
% 
Le relazioni indicate con "\textbf{*}" sono state modificate durante la ristrutturazione.
\\ 
\\
Durante la ristrutturazione sono state modificate le seguenti relazioni: \textbf{VideoFile}, \textbf{AudioFile}, \textbf{Regia}, \textbf{Recitazione}.
\\ \\ 
Mentre sono state aggiunte le seguenti: \textbf{Doppiaggio}, \textbf{Sottotitolaggio} \textbf{DistanzaPrecalcolata}, \textbf{Pagamento Carta}.
\\ \\
\begin{tabular}{|C{2cm}|C{4.5cm}|C{2.5cm}|C{1.9cm}|}
\hline
 \textbf{Relazione}& \textbf{Descrizione}& \textbf{Componenti}& \textbf{Attributi}\\ 
\hline
\hline
& & & \\
 \sethlcolor{green}\hl{VideoFile} & Presenza di un Formato Video in un File & FormatoCodifica (0,N), File (1,1) & FPS, Risoluzione \\ 
& & & \\
\hline
& & & \\
 \sethlcolor{green}\hl{AudioFile} & Presenza di un Formato Audio in un File & FormatoCodifica (0, N), File (1, 1) & Frequenza, BitDepth \\ 
& & & \\
\hline
& & & \\
 \sethlcolor{red}\hl{Recitazione} & Recitazione da parte di un Artista in un Film & Film (1, N), Artista (0, N) & - \\ 
& & & \\
\hline
& & & \\
 \sethlcolor{red}\hl{Regia} & Direzione da parte di un Artista di un Film & Artista (1, N), Film (1, 1) & - \\
& & & \\
\hline
& & & \\
 \sethlcolor{red}\hl{Doppiaggio} & Doppiaggio in una Lingua disponibile in un File & Lingua (0, N), File (1, N) & - \\
& & & \\
\hline
& & & \\
 \sethlcolor{red}\hl{Sottotitolaggio} & Sottotitolaggio in una Lingua disponibile in un File & Lingua (0, N), File (1, N) & - \\
& & & \\
\hline 
& & & \\
 \sethlcolor{yellow}\hl{DistanzaPrecalcolata} & Distanza Precalcolata tra un Paese e un Server & Paese (1, N), Server (1, N) & Valore Distanza \\
& & & \\
\hline 
& & & \\ 
 \sethlcolor{cyan}\hl{PagamentoCarta} & Corrispondenza tra un Pagamento e la Carta utilizzata & Pagamento (1, 1), Carta di Credito (1, N) & - \\
& & & \\
\hline
\end{tabular}
%
%
%
% Fine Pagina 12
%
% ~ ~ ~ ~ ~ ~ ~ ~ ~ ~ 
%
% Inizio Pagina 13
%
%
%
\subsection{Regole di Vincolo}
Le regole indicate con "\textbf{*}" sono state aggiunte durante la ristrutturazione. \\ \\
\begin{tabular}{|C{1.8cm}|C{6.5cm}|C{2.6cm}|}
\hline
& & \\
    \textbf{Area} & \textbf{Vincolo} & \textbf{Entità} \\
& & \\
\hline
    \sethlcolor{red}\hl{Contenuti}* & Un Artista deve aver recitato o diretto almeno un Film & \textbf{Film}, \textbf{Artista}, \textbf{Vincita} \\
\hline
    \sethlcolor{red}\hl{Contenuti}* & Una Lingua deve comparire in almeno un Doppiaggio o in almeno un Sottotitolaggio & \textbf{Lingua}, \textbf{Formato} \\
\hline
\end{tabular}

\subsection{Regole di Derivazione}
Le regole indicate con "\textbf{*}" sono state aggiunte durante la ristrutturazione. \\ \\
\begin{tabular}{|C{1.8cm}|C{6.5cm}|C{2.6cm}|}
\hline
& & \\
    \textbf{Area} & \textbf{Vincolo} & \textbf{Entità} \\
& & \\
\hline
    \sethlcolor{red}\hl{Contenuti} & Il rating critico di un Film è uguale alla media dei voti delle Critiche di quel film & \textbf{Film}, \textbf{Critico} \\
\hline
    \sethlcolor{red}\hl{Contenuti} & Il rating popolare di un Film è uguale alla media dei voti delle Recensioni di quel film ed è memorizzato come attributo ridondante nell'entità Film & \textbf{Film}, \textbf{Utente} \\
\hline
    \sethlcolor{yellow}\hl{Streaming}* & ValoreDistanza di DistanzaPrecalcolata è uguale alla distanza in km tra Posizione di Paese e Posizione di Server & \textbf{Server}, \textbf{Paese} \\
\hline
    \sethlcolor{yellow}\hl{Streaming} & CaricoAttuale di Server è una ridondanza che indica il numero di Erogazioni attive che impiegano tale Server. Quando una Erogazione viene avviata (inserita nel DB) il contatore aumenta di 1 e quando termina (viene cancellata) diminuisce di 1 & \textbf{Server}, \textbf{Visualizzazione} \\
\hline
    \sethlcolor{yellow}\hl{Streaming} & CaricoAttuale di Server deve essere sempre minore di MaxConnessioni, se così non è il Server è in sovraccarico & \textbf{Server} \\
\hline
\end{tabular}
%
%
%
% Fine Pagina 16 
%
% ~ ~ ~ ~ ~ ~ ~ ~ ~ ~ ~
%
% Inizio Pagina 17
%
%
%
\subsection{Descrizione ER}
In questo paragrafo sono trattati nel dettaglio alcuni attributi, entità o relazioni il cui significato o funzionamento potrebbe non risultare chiaro ed evidente guardando il diagramma Entità-Relazioni.

\subsubsection{Critico e Utente}
Ciascuna delle due entità ha la possibilità di esprimere un giudizio su un Film tramite un Voto. Un \textbf{Critico} però non è una persona qualunque: è un delegato di un giornale o un esperto di cinema e della sua \textbf{Critica} si memorizza anche un Testo: un commento testuale riguardante il \textbf{Film} criticato. Gli utenti che guardano un Film potranno leggere queste critiche prima di guardarlo e decidere quindi se iniziarne la visione o meno.
\\
\\
Entrambe le entità condividono inoltre tre attributi, motivo per cui è stata effettuata la generalizzazione (esclusiva), ma poiché le medie dei voti dei critici e quelle dei voti degli utenti saranno poi distinte, assieme al fatto che solo un Critico può allegare commenti è stato deciso, in fase di Ristrutturazione, di separare le entità. 

\subsubsection{Vincita e Premio}
Per \textbf{Macrotipo} e \textbf{Microtipo} si intendono, rispettivamente, la tipologia e la categoria del premio, ad esempio nel caso di "Oscar al miglior film" il \textbf{Macrotipo} sarebbe "Oscar", mentre il \textbf{Microtipo} "miglior film". \\
Per \textbf{Vincita} si intende un premio relativo ad uno specifico anno (e.g. Oscar al miglior film 2021)

\subsubsection{Edizione}
Uno stesso \textbf{Film} può essere ridistribuito più volte dalla casa di produzione e avere, per ogni nuova distribuzione piccoli o grandi cambiamenti, come l'aggiunta o la rimozione di scene. Un esempio in questo caso può essere \textit{Star Wars: Una Nuova Speranza} di George Lucas che, sebbene sia stato prodotto nel 1977 ha avuto decine di riedizioni (di cui l'ultima nel 2020); tra esse sono state effettuate piccole modifiche a dettagli o anche grandi a personaggi.
\\
Il Rapporto d'Aspetto è un altro attributo che può variare: uno stesso \textbf{Film} può averne vari, un esempio è il recente \textit{Oppenheimer} di Christopher Nolan, che è stato rilasciato sia in rapporto d'aspetto IMAX che in 16:9.
%
%
%
% Fine Pagina 17
%
% ~ ~ ~ ~ ~ ~ ~ ~ ~ ~ ~ ~ ~
%
% Inizio Pagina 18
%
%
%
\\
\\
Questo fa sì che la durata dello stesso \textbf{Film} possa cambiare tra le varie \textbf{Edizioni}: entità che ha lo scopo di raccogliere queste informazioni nella Base di Dati.
\\
\subsubsection{File}
Per \textbf{File} si intende un file in cui è stato salvato il contenuto di un edizione di un film. Esistono file, per una stessa \textbf{Edizione}, con risoluzione, dimensione e FPS differenti per ottemperare a varie necessità, ad esempio il limite di risoluzione per alcuni abbonamenti. \\ 
Con \textbf{FormatoContenitore} intendiamo l'estensione del file (e.g. .AVI .mp4). 
\subsubsection{FormatoCodifica}
Per \textbf{FormatoCodifica} si intende lo standard utilizzato per codificare dati audio e/o video (e.g. MPEG-4 Part 10 è lo standard più utilizzato dai file MP4) 
\subsubsection{Qualità di VideoFile}
La Qualità di un \textbf{File} Video viene misurata mediante due valori: Risoluzione e FPS (Frame Per second). La prima è la risoluzione (qualità d'immagine) dei fotogrammi, ovvero il numero di pixel che ciascun fotogramma possiede; il secondo è il numero di fotogrammi che verranno visualizzati ogni secondo.
 \\
Maggiori sono entrambi questi valori maggiore è la Qualità Video ma anche la dimensione del \textbf{File} e il suo BitRate. \\
\subsubsection{Qualità di AudioFile}
La Qualità di un File Audio viene misurata mediante due valori: Frequenza e BitDepth. La prima è la frequenza con cui il segnale audio è stato campionato (più alta è più la Qualità è maggiore); la seconda è la precisione con cui ogni punto è individuato (maggiore è il BitDepth maggiore è la Qualità del segnale).
\\ \\
Bisogna segnalare che, tuttavia, \textbf{File} con maggiore Qualità Audio hanno anche Dimensione maggiore e maggiore BitRate di fruizione. \\
%
%
%
% Fine Pagina 18
%
% ~ ~ ~ ~ ~ ~ ~ ~ ~ ~ ~ ~ ~ ~
%
% Inizio Pagina 19
%
%
%
\begin{figure}
    \centering
    \includegraphics[width=\linewidth,keepaspectratio]{assets/Campionamento segnale.png}
    \label{fig:Campionamento segnale spiegazione BitDepth e Frequenza}
\end{figure}
\subsubsection{IP Range}
Vi è la necessità di sapere da dove si connettono gli \textbf{Utenti} per avere una migliore fruizione dei contenuti ed anche applicare eventuli restrizioni dovute a leggi.
\\
\\
Se fosse l'utente stesso a comunicare il \textbf{Paese} (nazione) dove si trova ci sarebbe il problema di richeste al sistema con valori falsi: un \textbf{Utente} potrebbe dalle impostazioni del proprio dispositivo modificare sia la posizione riportata dal GPS (che diventa quindi inutile alla base di dati) che il \textbf{Paese} dove il dispositivo pensa di trovarsi.
\\
\\
L'utente non può però nascondere il suo indirizzo IP (senza usare una VPN) durante la connessione e da esso si può tramite questa entità risalire al \textbf{Paese} dove si trova.
\\
L'indirizzo IP di un dispositivo viene assegnato dal proprio ISP (Internet Service Provider, esempi sono TIM e Vodafone) al momento in cui ci si aggancia al ripetitore di quest'ultimo.
\\
\\
Gli indirizzi IP che un ISP può assegnare vengono comprati (e posseduti) in blocchi continui, individuabili quindi da un IP di inizio e da uno di Fine. La DataInizio di un range è la data in cui il range è stato acquistato ed è diventato quindi valido.
\\
\\
Ciascun ISP può avere sede in uno e un solo \textbf{Paese}, quindi un blocco di IP permette di risalire al \textbf{Paese} proprietario.
\\
\\
%
%
%
% Fine Pagina 19
%
% ~ ~ ~ ~ ~ ~ ~ ~ ~ ~ ~ 
%
% Inizio Pagina 20
%
%
%
Un blocco (o parte) di indirizzi può però essere rivenduto ad altri ISP e quindi cambiare \textbf{Paese} di riferimento.
\\
\\
Dei Vincoli di Integrità faranno in modo che nella Base di Dati non vi siano range che collidono entrambi validi in uno stesso istante temporale e faranno ciò assegnando una data di fine validità (DataFine) ai range quando ne viene inserito uno nuovo che collide con uno già presente e impostando come scaduti range ineriti con data di inizio validità inferiore ad altri con cui collidono già presenti
\\
\\
Due diversi range collidono se e solo se esiste un indirizzo IP che appartiene a entrambi i range.
\\
\begin{center}
    \centering
    \includegraphics[width=1\linewidth]{assets/InserimentoRange.png}
    % \caption{Esempio di inserimento di due IP-Range}
    \label{Il primo Range (Italia) viene svalidato quando viene inserito il secondo (Spagna) che si sovrappone ad esso}
\end{center}


\subsubsection{Server}
Il \textbf{CaricoAttuale} del server viene misurato in numero di connessioni attualmente aperte, mentre con \textbf{MaxConnessioni} ci riferiamo al massimo numero di connessioni che possono essere aperte in un certo istante. \\ 
Il server è considerato in sovraccarico se il numero di connnessioni attualmente aperte, ossia \textbf{CaricoAttuale}, è superiore ad una certa percentuale del numero massimo di connessioni (e.g. se CaricoAttuale supera l'80\% di MaxConnessioni il Server è in sovraccarico). \\ \\ 
Maximum Transmission Unit (\textbf{MTU}), in italiano Unità massima di trasmissione, si indica la dimensione massima , misurata in byte, di un pacchetto dati che può essere inviato.
%
%
%
% Fine Pagina 20
%
% ~ ~ ~ ~ ~ ~ ~ ~ ~ ~ ~ ~
%
% Inizio Pagina 21
%
%
%
\subsubsection{Abbonamento}
La \textbf{Durata} dell'\textbf{Abbonamento} è il numero di giorni dopo il quale l'abbonamento scade, insieme all'attributo \textbf{DataInizio} della relazione \textbf{AbbonamentoUtente} è possibile determinare se l'attuale abbonamento e' in vigore o scaduto. \\
Non viene salvato lo storico di tutti gli abbonamenti di un \textbf{Utente} ma solamente l'abbonamento più recente dell'utente, sia esso scaduto o in vigore. \\ \\ 
Con \textbf{Definizione} si intende la definizione massima consetita da quel tipo di abbonamento. \textbf{Offline} è un booleano che indica se l'abbonamento permette all'utente di scaricare contenuti in locale. \\
Con \textbf{MaxOre} ci si riferisce al numero massimo di ore mensili di cui un utente può usufruire per la fruizione di contenuti. \textbf{GB Mensili} sono il massimo numero di GB utilizzabili da un utente ogni mese.

\subsubsection{Popolarità di Artista}
La \textbf{Popolarità} di \textbf{Artista} è un valore numerico decimale che va da 0 a 10 e funge, per l'appunto, da metrica della fama, notorietà e celebrità di un artista. \\ 
Essa verrà, inoltre, presa in considerazione per fornire valutazioni riguardo i film in cui gli artisti prendono parte sia come attori che come registi.

\subsubsection{Posizione di un Paese}
La \textbf{Posizione} di un \textbf{Paese} è una coppia Latitudine, Longitudine (posizione sulla superficie terrestre) che identifica il "\textit{centro di popolazione}" di quest'ultimo. \\ 
Tale valore può essere calcolato come la media ponderata in base al numero di abitanti delle posizioni geografiche dei centri di un \textbf{Paese}. \\ \\
È necessario sfruttare questo valore fittizio e, di per sé, poco rappresentativo per individuare i \textbf{Server} migliori (più vicini) ad un \textbf{Utente} che si trova in un determinato \textbf{Paese}.

\newpage
\section{Ristrutturazione}

\subsection{Eliminazione delle Generalizzazioni}
\subsubsection{Artista}
\textbf{Artista} è una generalizzazione totale sovrapposta specializzata in \textbf{Attore} e \textbf{Regista} per distinguere i diversi ruoli che una persona può assumere all'interno di un film. 
\\ 
Entrambe le specializzazioni di \textbf{Artista} non sono dotate di attributi propri ma possiedono unicamente gli attributi relativi ad \textbf{Artista}. 
\\ 
\\
Questa generalizzazione è stata ristrutturata decidendo di conservare l'entità \textbf{Artista} la quale mantiene la relazione \textbf{VincitaArtista} ed, inoltre, guadagna le due relazioni \textbf{Recitazione} e \textbf{Regia} che erano precedentemente collegate, rispettivamente, a \textbf{Attore} e \textbf{Regista} cambiandone, tuttavia, le cardinalità. 
\\ 
\\
%
%
%
% Fine Pagina 21
%
% ~ ~ ~ ~ ~ ~ ~ ~ ~ ~ ~ ~ ~ ~ ~
%
% Inizio Pagina 22
%
%
%
Infatti, le cardinalità di \textbf{Regia} sono \textbf{(1, N)} rispetto a \textbf{Film}, come nell'ER non ristrutturato, e \textbf{(0, N)} verso \textbf{Artista}, diversamente da prima in quanto bisogna contemplare la presenza di un regista che non reciti in nessun film. Mentre le cardinalità di Recitazione sono \textbf{(1, 1)} rispetto a \textbf{Film}, come nell'ER non ristrutturato, e \textbf{(0, N)} verso \textbf{Artista}, diversamente da prima in quanto bisogna contemplare la presenza di un attore che non diregga nessun film. 
\\ 
\\ 
La nostra scelta di ristrutturazione deriva, innanzitutto, da una gestione sicuramente più efficiente del caso sporadico in cui esista un artista che ricopra sia un ruolo di \textbf{Attore} che un ruolo di \textbf{Regista} ed, oltretutto, dalla consapevolezza che non vi sia uno spreco di memoria, che si sarebbe verificato qualora avessimo ristrutturato conservando sia \textbf{Attore} che \textbf{Regista}, dovuto alla natura sovrapposta della generalizzazione. 
\\ 
\\
\subsubsection{Recensore}
\textbf{Recensore} è una generalizzazione totale ed esclusiva specializzata in \textbf{Critico} e \textbf{Utente} per distinguere le diverse valutazioni, ossia critica e recensione, che vengono espresse nei confronti di un film. 
\\
La specializzazione \textbf{Critico} non è dotata di attributi propri, mentre la specializzazione \textbf{Utente} possiede due attributi propri, ossia email e password. 
\\
Questa generalizzazione è stata ristrutturata decidendo di conservare le due entità \textbf{Critico} ed \textbf{Utente} eliminando l'entità \textbf{Recensore}. 
\\
\\
La nostra scelta di ristrutturazione deriva, innanzitutto, da una gestione indubbiamente più efficiente della memoria giacché, ristrutturando solamente con l'entità \textbf{Recensore}, ci sarebbero valori NULL, sebbene sporadici, per email e password nel caso dei critici ed, oltretutto, dalla chiara separazione sia a livello concettuale e logico ma anche pratico e funzionale tra un \textbf{Critico}, che fornisce unicamente critiche, ed un utente il quale, oltre a recensire, ha un evidente ruolo nell'\textbf{Area Clienti}, ovvero tutto ciò che riguarda abbonamenti, visualizzazioni e pagamenti.
\\
\subsubsection{FormatoCodifica}
\textbf{FormatoCodifica} è una generalizzazione totale e sovrapposta specializzata in \textbf{Audio} e \textbf{Video} per distinguere i formati utilizzati codificazione e decodificazione di dati uditivi e/o visivi. 
\\ 
Entrambe le specializzazioni di \textbf{FormatoCodifica} non sono dotate di attributi propri ma possiedono unicamente gli attributi relativi a \textbf{FormatoCodifica}. 
\\ 
%
%
%
% Fine Pagina 22 
%
% ~ ~ ~ ~ ~ ~ ~ ~ ~ ~ ~ ~
%
% Inizio Pagina 23 
%
%
%
Questa generalizzazione è stata tradotta decidendo di mantenere solamente l'entita \textbf{FormatoCodifica} la quale preserva le due relazioni con \textbf{File}, ossia \textbf{AudioFile} e \textbf{VideoFile}, modificandone, però, le cardinalità. 
\\ \\ 
Infatti, le cardinalità di \textbf{VideoFile} rimangono \textbf{(1,1)} rispetto a \textbf{File} ma mutano in \textbf{(0, N)} rispetto a \textbf{FormatoCodifica} poiché bisogna considerare l'esistenza di un formato unicamente audio. Invece, le cardinalità di \textbf{AudioFile} rimangono \textbf{(1,1)} rispetto a \textbf{File} ma mutano in \textbf{(0, N)} rispetto a \textbf{FormatoCodifica} poiché bisogna considerare l'esistenza di un formato unicamente video.
\\ \\ 
La nostra decisione di ristrutturazione è dovuta sia alla semplicità di introdurre una sola entità con una relazione piuttosto che due entità con due relazione, senza nessuno spreco di memoria a cagione dell'assenza di attributi annessi alle singole specializzazioni, che alla gestione più efficiente del caso in cui un \textbf{FormatoCodifica} sia sia \textbf{Audio} che \textbf{Video}, essendo la generalizzazione sovrapposta, la quale non comporterebbe alcuno spreco di memoria.
\newpage
%
%
%
% Fine Pagina 23
%
% ~ ~ ~ ~ ~ ~ ~ ~ ~ ~ ~
%
% Inizio Pagina 24
%
%
%
\subsection{Eliminazione degli Attributi Multivalore}
\subsubsection{LinguaDoppiaggio}
Si considera che ogni \textbf{File} debba contenere al suo interno audio in una o più lingue. L'attributo multivalore è, pertanto, stato gestito introducendo sia un'entità \textbf{Lingua}, usata altresì per le lingue in cui sono disponibili i sottotitoli, che una relazione \textbf{Doppiaggio} che intercorre tra \textbf{Lingua}, con cardinalità \textbf{(0, N)}, in quanto una lingua potrebbe essere disponibile nei sottotitoli di qualche film ma mai disponibile come audio in nessun film, e \textbf{File}, con cardinalità \textbf{(1, N)} in quanto ogni file deve avere l'audio disponibile in almeno una lingua (si ignora la presenza del cinema muto in quanto ormai desueto).
\subsubsection{LinguaSottotitoli}
Si considera che ogni \textbf{File} possa contenere al suo interno \textbf{Sottotitoli} in varie lingue. L'attributo multivalore è stato trasformato in una relazione \textbf{Sottotitolaggio} che intercorre tra \textbf{Lingua} (creata al punto precedente), con cardinalità \textbf{(0, N)}, in quanto una lingua può essere presente come audio in un film e mai nei sottotitoli di alcun film, e \textbf{File}, con cardinalità \textbf{(0, N)}, in quanto un film non deve necessariamente avere sottotitoli.
\subsection{Analisi Ridondanze}
\subsubsection{Ridondanze presenti}
Nell'ER iniziale è presente una sola ridondanza: \textit{CaricoAttuale} di \textbf{Server}.
\\
Essa è stata inserita in modo da facilitare (velocizzare) l'individuazione dei server sovraccarichi e, se ve ne sono, operare sulle connessioni attive su quei server per bilanciare la distribuzione degli utenti. Per garantire una fluidità del servizio questa operazione è ripetuta con alta frequenza e deve quindi completare velocemente in modo da non sovraccaricare lei stessa il DB.
\\
Essa è calcolata contando tutte le \textbf{Erogazioni} attive (presenti nel DB) sul server
\subsubsection{Introduzione nuove Ridondanze}
È stata inserita una ulteriore ridondanza sotto la forma della relazione \textbf{DistanzaPrecalcolata} tra \textbf{Server} e \textbf{Paese}, che conterrà all'interno la distanza (espressa in kilometri) tra ogni \textbf{Server} e ogni \textbf{Paese}, in modo da velocizzare l'esecuzione della query che restituisce il server più adatto per la fruizione di un contenuto da parte dell'utente evitando il ricalcolo della distanza tra ogni \textbf{Server} ed il \textbf{Paese} dell'utente.
\\ \\
Sono stati aggiunte all'entità \textbf{Film} due attributi ridondanti: \textit{MediaRecensioni} e \textit{NumeroRecensioni}. \\
Essi sono necessari per evitare di calcolare la media delle recensioni di ogni \textbf{Film}, operazione molto comune perché avviene ogni volta che un \textbf{Utente} apre la pagina relativa ad un \textbf{Film}.\\
\textit{NumeroRecensioni} permette di modificare la \textit{MediaRecensioni} alla aggiunta / modifica / rimozione di una \textbf{Recensione} senza ricalcolare il valore dall'intero insieme dei dati.
%
%
%
% Fine Pagina 24
%
% ~ ~ ~ ~ ~ ~ ~ ~ ~ ~ ~
%
% Inizio Pagina 25
%
%
%
\subsection{Partizionamento e Accorpamento ER}
\subsubsection{Qualità di VideoFile}
L'attributo composto \textbf{Qualità} non è altro che un raggruppamento di specifiche tecniche nell'ambito video di un \textbf{File} multimediale; è quindi scomposto in due attributi \textit{Risoluzione} e \textit{FPS}. 
\subsubsection{Qualità di AudioFile}
L'attributo composto \textbf{Qualità} non è altro che un raggruppamento di specifiche tecniche nell'ambito audio di un \textbf{File} multimediale; è quindi scomposto in due attributi \textit{Frequenza} e \textit{BitDepth}. 
\subsubsection{Posizione}
L'attributo composto \textbf{Posizione} descrive una coppia Latitudine e Longitudine in modo da individuare un punto sulla superficie terrestre.
\\
Esso, a differenza degli altri attributi composti, non necessita di essere scomposto per facilitare l'Implementazione Fisica, in quanto il DBMS in cui la Base di Dati sarà implementata (Oracle MySQL) supporta attributi di tipo \textit{POINT}, che possono essere utilizzati per memorizzare coppie di valori numerici assieme.
\\
Oltretutto, memorizzare una coppia \textit{Latitudine} e \textit{Longitudine} in questo modo facilita successivamente il calcolo di distanza tra due \textbf{Posizioni}, sfruttando la funzione nativa di MySQL \textit{ST\_DISTANCE\_SPHERE}.
\\
\subsubsection{Carta di Credito}
L'attributo composto \textbf{Carta di Credito}, dell'entità \textbf{Pagamento} contiene le informazioni sulla carta impegata nel pagamento di una \textbf{Fattura}.
\\
\\
%
%
%
% Fine Pagina 25
%
% ~ ~ ~ ~ ~ ~ ~ ~ ~ ~ ~ ~
%
% Inizio Pagina 26
%
%
%
Essa è stata trasformata in un'entità, poiché si prevede che un \textbf{Utente} utilizzi più volte la stessa carta; pertanto, se tutti i dati della carta fossero attributi di \textbf{Pagamento} non vi sarebbero una gestione efficiente della memoria a causa di valori ripetuti.
\\
È stata quindi creata l'entità \textbf{Carta di Credito} con i seguenti attributi: \textit{PAN}, \textit{Scadenza}, \textit{CVV}. La chiave dell'entità è \textit{PAN}, in quanto ogni carta può essere individuata unicamente dal suo numero.
\\
\\
È stata quindi aggiunta anche una relazione \textbf{PagamentoCarta} tra le entità \textbf{Pagamento}, con cardinalità \textbf{(1, 1)}, e \textbf{Carta di Credito}, con carinalità \textbf{(1, N)}, in quanto una carta può essere usata in più pagamenti ed in un pagamento può essere usata una sola carta.

\newpage
%
%
%
% Fine Pagina 26
%
% ~ ~ ~ ~ ~ ~ ~ ~ ~ ~ ~ ~
%
% Inizio Pagina 27
%
%
%
\section{Tavole dei Volumi}


Le entità e le relazioni con segante con "*" hanno volume assegnato per ipotesi.

\subsection{Area Contenuti}
\begin{tabular}{|C{2.3cm}|C{0.7cm}|C{1.7cm}|C{6.2cm}|}
\hline
    \textbf{Concetto} & \textbf{Tipo} & \textbf{Volume} & \textbf{Note} \\
\hline
\hline
& & & \\
    Film* & E & 1\,000 & Assunzione iniziale \\
& & & \\
\hline
& & & \\
    Produzione & R & 1\,000 & Cardinalità (1, 1) con \textbf{Film} \\
& & & \\
\hline
& & & \\    
Casa di Produzione* & E & 20 & Assunzione iniziale \\
& & & \\
\hline
& & & \\    
    Appar-tenenza & R & 20 & Cardinalità (1, 1) con \textbf{Casa di Produzione} \\
& & & \\
\hline
& & & \\    
    Paese & E & 195+1 & Nel mondo ci sono 195 stati riconosciuti, il 196-esimo rappresenta lo stato \textit{Mondo}, utilizzato come ancora di salvezza per dare una posizione agli utenti quando non si riesce a trovare lo stato in cui si trovano \\
& & & \\
\hline
& & & \\    
    Lingua & E & 141 & Nel mondo ci sono 141 lingue \\
& & & \\
\hline
& & & \\    
    Doppiaggio & R & 18 000 & Consideriamo che ogni \textbf{File} sia dobbiato, in media, in 4 \textbf{Lingue} \\
& & & \\
\hline
& & & \\    
    Sotto-titolaggio & R & 13 500 & Consideriamo che ogni \textbf{File} sia sottotitolato, in media, in 3 \textbf{Lingue} \\
& & & \\
\hline
\end{tabular}
\\ \\ \\ \\
%
%
%
% Fine Pagina 27
%
% ~ ~ ~ ~ ~ ~ ~ ~ ~ ~ ~
%
% Inizio Pagina 28
%
%
%
\begin{tabular}{|C{2.3cm}|C{0.7cm}|C{1.7cm}|C{6.2cm}|}
\hline
    \textbf{Concetto} & \textbf{Tipo} & \textbf{Volume} & \textbf{Note} \\
\hline
\hline
& & & \\    
    Edizione Film & R & 1 500 & Cardinalità (1, 1) con \textbf{Edizione} \\
& & & \\
\hline
& & & \\    
    Artista* & E & 4 000 & Assunzione Iniziale \\
& & & \\
\hline
& & & \\    
    Regia & R & 1 000 & Cardinalità (1, 1) con \textbf{Film} \\
& & & \\
\hline
& & & \\    
    Recitazione & R & 20 000 &  Consideriamo che, in media, 20 attori recitino in un \textbf{Film} \\
& & & \\
\hline
& & & \\    
    Genere* & E & 27 & Assunzione iniziale \\
& & & \\
\hline
& & & \\    
    GenereFilm & R & 2 000 & Consideriamo che, in media, un \textbf{Film} appartenga a due \textbf{Generi} \\
& & & \\
\hline
& & & \\     
    Critico* & E & 25 & Assunzione iniziale \\
& & & \\
\hline
& & & \\    
    Critica & R & 400 & Si stima che un \textbf{Critico} faccia in media poco più di 15 critiche a \textbf{Film} diversi \\
& & & \\
\hline
& & & \\    
    Recensione & R & 25 000 & Si stima che in media un \textbf{Film} abbia 25 recensioni da \textbf{Utenti} diversi \\
& & & \\
\hline
& & & \\    
    Premio* & E & 50 & Assunzione iniziale \\
& & & \\
\hline 
\end{tabular}
\begin{tabular}{|C{2.3cm}|C{0.7cm}|C{1.7cm}|C{6.2cm}|}
\hline
    \textbf{Concetto} & \textbf{Tipo} & \textbf{Volume} & \textbf{Note} \\
\hline
\hline
& & & \\    
    Vincita & E & 1 500 & Si stima che ogni \textbf{Premio} sia stato vinto una volta l'anno negli ultimi 30 anni \\
& & & \\
\hline
& & & \\    
    VincitaPremio & R & 1 500 & Cardinalità (1, 1) con \textbf{Vincita} \\
& & & \\
\hline
& & & \\    
    VincitaFilm & R & 1 500 & Cardinalità (1, 1) con \textbf{Vincita} \\
& & & \\
\hline
& & & \\    
    VincitaArtista & R & 1 200 & Si stima che la maggior parte (80\%) delle vincite riguardi in maniera specifica un \textbf{Artista} (Attore o Regista)  \\
& & & \\
\hline 
\end{tabular}
\subsection{Area Formato}
\begin{tabular}{|C{2.3cm}|C{0.7cm}|C{1.7cm}|C{6.2cm}|}
\hline
    \textbf{Concetto} & \textbf{Tipo} & \textbf{Volume} & \textbf{Note} \\
\hline
\hline
& & & \\ 
    Edizone & E & 1 500 & Si stima che, in media, la metà dei \textbf{Film} abbia due edizioni \\
& & & \\
\hline
& & & \\    
    File & E & 4 500 & Si stima che ogni \textbf{Edizione} abbia in media 3 \textbf{File} associati, in modo da permettere la fruizione su dispositivi diversi (che possono supportare solo alcuni formati) e anche velocità di connessione deboli \\
& & & \\
\hline
& & & \\    
    Edizone File & R & 4 500 & Cardinalità (1, 1) con \textbf{File} \\
& & & \\
\hline
\end{tabular} \\ \\
\begin{tabular}{|C{2.3cm}|C{0.7cm}|C{1.7cm}|C{6.2cm}|}
\hline
    \textbf{Concetto} & \textbf{Tipo} & \textbf{Volume} & \textbf{Note} \\
\hline
\hline
& & & \\    
    Formato Codifica* & E & 100 & Si stima che esistano circa 100 Formati di codifica tra video e audio \\
& & & \\
\hline
& & & \\    
    VideoFile & R & 4 500 & Cardinalità (1, 1) con \textbf{File} \\
& & & \\
\hline
& & & \\    
    AudioFile & R & 4 500 & Cardinalità (1, 1) con \textbf{File} \\
& & & \\
\hline
& & & \\    
    Restrizione & R & 1 000 &  Si stima che ogni \textbf{Paese} bandisca, in media, 5 \textbf{Edizioni} \\
& & & \\ 
\hline
\end{tabular} \\
\subsection{Area Clienti}
\begin{tabular}{|C{2.3cm}|C{0.7cm}|C{1.7cm}|C{6.2cm}|}
\hline
    \textbf{Concetto} & \textbf{Tipo} & \textbf{Volume} & \textbf{Note} \\
\hline
\hline
& & & \\
    Utente* & E & 1 000 000 & Assunzione iniziale \\
& & & \\
\hline
& & & \\    
    Abbon-Utente & R & 1 000 000 & Cardinalità (1, 1) con \textbf{Utente} \\
& & & \\
\hline
& & & \\    
    Abbonamento* & E & 5 & Assunzione iniziale \\
& & & \\
\hline
& & & \\    
    Esclusione & R & 15 & Si stima che ogni \textbf{Abbonamento} escluda, in media, circa 3 \textbf{Generi} \\
& & & \\
\hline
& & & \\    
    Fattura & E & 12 000 000 & Fattura contiene le fatture emesse in un anno. Per ogni utente viene emessa una fattura ogni mese, quindi 12 in un anno. \\
& & & \\
\hline
\end{tabular} \\ \\
\begin{tabular}{|C{2.3cm}|C{0.7cm}|C{1.7cm}|C{6.2cm}|}
\hline
    \textbf{Concetto} & \textbf{Tipo} & \textbf{Volume} & \textbf{Note} \\
\hline
\hline
& & & \\    
    Emissione & R & 12 000 000 & Cardinalità (1, 1) con \textbf{Fattura} \\
& & & \\
\hline
& & & \\    
    Pagamento & E & 11 000 000 & Si stima che per ogni utente le 11 ultime fatture siano state pagate mentre l'ultima no. Quindi 11 su 12, ovvero il 92\% delle emissioni.  \\
& & & \\
\hline
& & & \\    
    Pagamento Fattura & R & 11 000 000 & Cardinalità (1, 1) con \textbf{Pagamento} \\
& & & \\
\hline
& & & \\    
    Carta di credito & E & 1 500 000 & Si stima che la metà degli \textbf{Utenti} abbia usato, in media,  due \textbf{Carte di Credito} \\
& & & \\
\hline
& & & \\    
    Pagamento Carta & R & 11 000 000 & Cardinalità (1, 1) con \textbf{Pagamento} \\
& & & \\
\hline
& & & \\    
    Connessione & E & 30 000 000 & Si stima che un \textbf{Utente} si connetta 1 volta al giorno. Il sistema conserva le connessioni degli ultimi 30 giorni \\
& & & \\
\hline
& & & \\    
    Conn-Utente & R & 30 000 000 & Cardinalità (1, 1) con \textbf{Connessione} \\
& & & \\
\hline
& & & \\    
    Visualizzazione & E & 15 000 000 & Si stima che un \textbf{Utente} guardi, in media, un'\textbf{Edizione} ogni due giorni. Il sistema tiene traccia dei contenuti guardati negli ultimi 30 giorni \\
& & & \\
\hline
& & & \\    
    Visualizzazione Connessione & R & 15 000 000  & Cardinalità (1, 1) con \textbf{Visualizzazione} \\
& & & \\
\hline
\end{tabular}
%
%
%
% Fine Pagina 29
%
% ~ ~ ~ ~ ~ ~ ~ ~ ~ ~ ~
%
% Inizio Pagina 30
%
%
%
\begin{tabular}{|C{2.3cm}|C{0.7cm}|C{1.7cm}|C{6.2cm}|}
\hline
    \textbf{Concetto} & \textbf{Tipo} & \textbf{Volume} & \textbf{Note} \\
\hline
\hline
& & & \\    
    Visualizzazione Edizione & R & 15 000 000 & Cardinalità (1, 1) con \textbf{Visualizzazione} \\
& & & \\
\hline
& & & \\    
    Visualizzazione Edizione & R & 15 000 000 & Cardinalità (1, 1) con \textbf{Visualizzazione} \\
& & & \\
\hline
\end{tabular}
\subsection{Area Streaming}
\begin{tabular}{|C{2.3cm}|C{0.7cm}|C{1.7cm}|C{6.2cm}|}
\hline
    \textbf{Concetto} & \textbf{Tipo} & \textbf{Volume} & \textbf{Note} \\
\hline
\hline
& & & \\
    Server* & E & 20 & Circa il 10\% dei \textbf{Paesi} \\
& & & \\
\hline
& & & \\    
    Distanza Precalcolata & R & 3920 & Esattamente ogni combinazione possibile tra \textbf{Server} e \textbf{Paese} \\
& & & \\
\hline
& & & \\    
    P.o.P. & R & 9 000 & Ogni file è presente, in media, su 2 server \\
& & & \\
\hline
& & & \\    
    Erogazione & R & 40 000 & Corrisponde al numero di utenti mediamente connessi in ogni istante. Stimando il numero medio di ore giornaliere di attività a 1 si deduce che il 4\% degli utenti è mediamente connesso in ogni istante \\
& & & \\
\hline
& & & \\    
    Ip-Range* & E & 130 000 & Assunzione iniziale basata su valori reali \\
& & & \\
\hline
& & & \\    
    GeoLocal & R & 130 000 & Cardinalità (1, 1) con Ip-Range \\
& & & \\
\hline
\end{tabular}
\section{Operazioni sui Dati}
\subsection{Analisi delle Ridondanze}
\subsubsection{CaricoAttuale}
\textit{CaricoAttuale}, attributo di \textbf{Server} contiene, in ogni momento, il numero di connessioni in corso di un server CDN. Tale valore potrebbe essere calcolato contando le \textbf{Erogazioni} che puntano a quel \textbf{Server}. Per misurare il carico di un singolo server, senza la ridondanza, sarebbe necessario quindi analizzare tutte le Erogazioni e contare quelle che soddisfano il criterio. \\
Tale operazione avviene però quando un \textbf{Utente} è alla ricerca del \textbf{Server} migliore che gli può offrire il contenuto che cerca, rendendo necessario analizzare tutti il CaricoAttuale di tutti i Server.\\
Questa operazione è molto comune e deve concludersi in fretta in modo da non far aspettare l'Utente e garantire una migliore esperienza.
\\ \\ 
Il tipo di aggiornamento della ridondanza è l'\textbf{Immediate Refresh}: essa viene aggiornata ad ogni inserimento/modifica/cancellazione nella relazione \textbf{Erogazione}.
\\ \\
%
%
%
% Fine Pagina 30
%
% ~ ~ ~ ~ ~ ~ ~ ~ ~ ~ ~
%
% Inizio Pagina 31
%
%
%
Di seguito è riportata la frequenza giornaliera delle attività che coinvolgono tale ridondanza: \\ \\
\begin{tabular}{|C{2.2cm}|C{3.8cm}|C{4.4cm}|}
\hline
    \textbf{Azione} & \textbf{FrequenzaGiornaliera} & \textbf{Nota} \\
\hline
Lista Server & 500 000 & Si considera che ogni utente guardi un file ogni due giorni \\
\hline
\end{tabular}
\\ \\ 
\textbf{Input}: chiave del \textbf{File} che si intende guardare \\
\textbf{Output}: Lista di Server disponibili, ciascuno con il suo CaricoAttuale
\\ \\
I Volumi interessati sono: \\ \\ 
\begin{tabular}{|C{2.6cm}|C{2.6cm}|C{2.6cm}|}
\hline
    \textbf{Nome} & \textbf{Tipo} & \textbf{Volume} \\
\hline
 P.o.P & E & 9 000 \\
\hline
 Erogazione & R & 40 000 \\
\hline
\end{tabular}
\\ \\ \\ 
Gli accessi per Visualizzazione sarebbero in media, senza ridondanza: \\ \\
\begin{tabular}{|C{1.7cm}|C{1.7cm}|C{1.3cm}|C{0.8cm}|C{5.1cm}|}
\hline
    \textbf{Concetto} & \textbf{Costrutto} & \textbf{Accessi} & \textbf{Tipo} & \textbf{Note} \\
\hline
    P.o.P & R & 9 000 & R & Conoscendo la chiave di \textbf{File} si può passare direttamente a leggere i \textbf{P.o.P.} per individuare i \textbf{Server} che possiedono il \textbf{File} che si cerca. La lettura di tutte le occorrenze della relazione restituirà, in media, 2 chiavi di \textbf{Server} \\
\hline
    Erogazione & R & 40 000 & R & Vengono lette tutte le occorrenze per individuare quelle che soddisfano il criterio. La chiave si \textbf{Server} è stata ottenuta da \textbf{P.o.P.} \\
\hline
    \multicolumn{2}{|c|}{\textbf{Totale:}} & 
    \multicolumn{2}{c|}{49 000} & 
    Soluzione non ottimale \\ 
\hline
\end{tabular} \\ \\ \\ \\ 
Mentre, con la ridondanza sono in media: \\ \\
%
%
%
% Fine Pagina 31
%
% ~ ~ ~ ~ ~ ~ ~ ~ ~ ~ ~
%
% Inizio Pagina 32
%
%
%
\begin{tabular}{|m{1.7cm}|m{1.7cm}|m{1.3cm}|m{0.8cm}|m{5.1cm}|}
\hline
    \textbf{Concetto} & \textbf{Costrutto} & \textbf{Accessi} & \textbf{Tipo} & \textbf{Note} \\
\hline
    P.o.P & R & 9 000 & R & Conoscendo la chiave di \textbf{File} si può passare direttamente a leggere i \textbf{P.o.P.} per individuare i \textbf{Server} che possiedono il \textbf{File} che si cerca. La lettura di tutte le occorrenze della relazione restituirà, in media, 2 chiavi di \textbf{Server} \\
\hline
    Server & E & 2 & R & La chiave di \textbf{Server} è già stata individuata e accedendo all'entità si ha già il CaricoAttuale: parametro della scelta \\
\hline
    \multicolumn{2}{|c|}{\textbf{Totale:}} & 
    \multicolumn{2}{c|}{9 002} & 
    Soluzione ottimale \\ 
\hline
\end{tabular}
\\ \\ \\ 
È importante far notare che, in questo caso, ogni volta che un'\textbf{Erogazione} viene aggiunta/rimossa sarà necessario accedere a \textbf{Server} per alterarne il CaricoAttuale. Questo costringe a sommare ai valori giornalieri ottenuti il costo ottenuto dalla periodica modifica di CaricoAttuale. \\ \\ 
\begin{tabular}{|C{2.2cm}|C{3.8cm}|C{4.4cm}|}
\hline
    \textbf{Azione} & \textbf{FrequenzaGiornaliera} & \textbf{Nota} \\
\hline
Aggiornamento \textit{CaricoAttuale} & 750 000 & Si considera che ogni \textbf{Visualizzazione} sfrutti, in media, 1,5 \textbf{Server} diversi nel corso della sua durata \\
\hline
\end{tabular} \\ \\ \\
I Volumi coinvolti sono: \\ \\ 
\begin{tabular}{|C{2.6cm}|C{2.6cm}|C{2.6cm}|}
\hline
    \textbf{Nome} & \textbf{Tipo} & \textbf{Volume} \\
\hline
 Server & E & 20 \\
\hline
\end{tabular} \\ \\ 
La cui tabella degli Accessi è: \\ \\
\begin{tabular}{|C{1.7cm}|C{1.7cm}|C{1.3cm}|C{0.8cm}|C{5.1cm}|}
\hline
    \textbf{Concetto} & \textbf{Costrutto} & \textbf{Accessi} & \textbf{Tipo} & \textbf{Note} \\
\hline
    Server & E & 2 & W & La chiave di \textbf{Server} è già nota alla relazione \textbf{Erogazione}, permettendo di accedere al \textbf{Server} interessato e impostare il \textit{CaricoAttuale} a +1 o -1 il suo precedente valore se l'\textbf{Erogazione} è stata aggiunta o rimossa. \\
\hline
    \multicolumn{2}{|c|}{Totale:} & \multicolumn{2}{c|}{1} & \\
\hline
\end{tabular}
%
%
%
% Fine Pagina 32
%
% ~ ~ ~ ~ ~ ~ ~ ~ ~ ~ ~
%
% Inizio Pagina 33
%
%
%
\\ \\ \\
Gli accessi giornalieri delle due tipologie sono quindi: \\ \\
\begin{tabular}{|C{3cm}|C{2.5cm}|C{2.5cm}|C{2cm}|}
\cline{2-4}
   \multicolumn{1}{c}{}  & \multicolumn{3}{|c|}{Accessi al giorno} \\
\hline
    \textbf{Metodologia} & \textbf{Lettura} & \textbf{Scrittura} & \textbf{Totale} \\
\hline
    Senza ridondanza & 24,5 X $ 10^{9} $ & 0 & 24,5 X $ 10^{9} $ \\
\hline
    Con ridondanza & 4,5 X $ 10^{9} $ & 500 000 & 4,5 X $ 10^{9} $\\
\hline
\end{tabular}
\\ \\ 
% Problema relativo al modo di contare gli accessi
\subsubsection{DistanzaPrecalcolata}
\textbf{DistanzaPrecalcolata} è una relazione tra \textbf{Paese} e \textbf{Server} che ha, come attributo, il valore della distanza tra un server ed un paese. Tale valore potrebbe essere calcolato a partire dalle posizioni sia di paese che di server. \\
Qualora si stia ricercando il server ottimale per un utente ubicato in un certo paese occorre conoscere, oltre al carico del server, anche la distanza tra di esso ed il paese. Invece di ricalcolare tutte le distanze ogni volta è più efficiente andarle a leggere da \textbf{DistanzaPrecalcolata}. \\ \\
\textbf{DistanzaPrecalcolata} non viene aggiornata giornalmente in quanto la posizione dei server e dei paesi è fissa.
\\ \\
Di seguito è riportata la frequenza giornaliera delle attività che coinvolgono tale ridondanza: \\ \\
\begin{tabular}{|C{2.2cm}|C{3.8cm}|C{4.4cm}|}
\hline
    \textbf{Azione} & \textbf{FrequenzaGiornaliera} & \textbf{Nota} \\
\hline
Distanza Server & 1 000 000 & Si considera che ogni \textbf{Utente} guardi un file ogni due giorni, e che per farlo usi circa due \textbf{Server}, dei quali si deve calcolare la distanza dalla posizione dell'\textbf{Utente}. Per ogni richiesta si stima che ci siano due \textbf{Server} (di cui si conosce già la chiave) che ospitano il contenuto richiesto \\
\hline
\end{tabular}
\\ \\ 
\textbf{Input}: chiave del \textbf{Paese} dove si trova l'utente e dei Server (2 in media) che ospitano il Film richiesto\\
\textbf{Output}: Lista di ID di tutti i Server, ciascuno con la sua distanza dal paese dell'utente
\\ \\
%
%
%
% Fine Pagina 33
%
% ~ ~ ~ ~ ~ ~ ~ ~ ~ ~ ~
%
% Inizio Pagina 34
%
%
%
I Volumi interessati sono: \\ \\ 
\begin{tabular}{|C{2.6cm}|C{2.6cm}|C{2.6cm}|}
\hline
    \textbf{Nome} & \textbf{Tipo} & \textbf{Volume} \\
\hline
 Distanza Precalcolata & R & 3 920 \\
\hline
\end{tabular}
\\ \\ \\ 
Gli accessi per Distanza Server sarebbero in media, senza ridondanza: \\ \\
\begin{tabular}{|C{1.7cm}|C{1.7cm}|C{1.3cm}|C{0.8cm}|C{5.1cm}|}
\hline
    \textbf{Concetto} & \textbf{Costrutto} & \textbf{Accessi} & \textbf{Tipo} & \textbf{Note} \\
\hline
    Paese & E & 1 & R & Conoscendo la chiave di \textbf{Paese} vi si può accedere puntualmente \\
\hline
    Server & E & 2 & R & Viene letta la posizione dei \textbf{Server} richiesti \\
\hline
    \multicolumn{2}{|c|}{\textbf{Totale:}} & 
    \multicolumn{2}{c|}{3} & 
    Soluzione non ottimale \\ 
\hline
\end{tabular} \\ \\ \\ \\ 
Mentre, con la ridondanza sono in media: \\ \\
\begin{tabular}{|C{1.7cm}|C{1.7cm}|C{1.3cm}|C{0.8cm}|C{5.1cm}|}
\hline
    \textbf{Concetto} & \textbf{Costrutto} & \textbf{Accessi} & \textbf{Tipo} & \textbf{Note} \\
\hline
    Distanza- Precalcolata & R & 2 & R & Conoscendo sia la chiave di \textbf{Paese} che di \textbf{Server} si può leggere la relazione in maniera puntuale.\\
\hline

    \multicolumn{2}{|c|}{\textbf{Totale:}} & 
    \multicolumn{2}{c|}{2} & 
    Soluzione ottimale \\ 
\hline
\end{tabular} \\ \\
Questa soluzione è ottimale anche perché i valori di \textit{Posizione} sia di un \textbf{Paese} che di un \textbf{Server} sono praticamente immutabili: è un evento rarissimo che una nazione nasca/scompaia o cambi notevolmente i suoi confini, come un \textbf{Server} non può cambiare posizione mentre sta erogando contenuti. Le distanze tra le \textit{Posizioni} possono quindi essere calcolate una volta sola all'inserimento di queste entità e praticamente mai cambiate. \\
Oltretutto, calcolare la distanza date le due posizioni possiede un costo computazionale non indifferente che, tuttavia, non può essere stimato in accessi. \\ \\
Gli accessi giornalieri delle due tipologie sono quindi: \\ 
\begin{tabular}{|C{3cm}|C{2.5cm}|C{2.5cm}|C{2cm}|}
\cline{2-4}
   \multicolumn{1}{c}{}  & \multicolumn{3}{|c|}{Accessi al giorno} \\
\hline
    \textbf{Metodologia} & \textbf{Lettura} & \textbf{Scrittura} & \textbf{Totale} \\
\hline
    Senza ridondanza & 3 X $ 10^{6} $ & 0 & 3 X $ 10^{6} $ \\
\hline
    Con ridondanza & 2 X $ 10^{6} $ & 0 & 2 X $ 10^{6} $\\
\hline
\end{tabular}
%
%
%
% Fine Pagina 34
%
% ~ ~ ~ ~ ~ ~ ~ ~ ~ ~ ~
%
% Inizio Pagina 35
%
%
%
\subsubsection{MediaRecensioni NumeroRecensioni}
\textbf{MediaRecensioni} e \textbf{NumeroRecensioni} sono due attributi di \textbf{Film} che tengono conto, rispettivamente, della media delle recensioni del film e del numero di recensioni che esso ha ricevuto. \\
Il valore della media ha un'utilità pratica e immediata mentre il numero di recensione assolve alla funzione di alleggerimento del mantenimento della ridondanza. \\ \\ 
Il valore della media delle recensione potrebbe essere calcolando andando a considerare tutte le recensioni del film e, poi, calcolandone la media. Qualore, però, si stia ricercando il film con valutazione media maggiore, la quale è un'operazione piuttosto comune, si andrebbe a ricalcolare molteplici volte le medie di tutti i film. \\
Invece di ricalcolarle ogni volta ex nihilo è più efficiente salvarle per poi rileggerle. 
\\ \\ 
Il tipo di aggiornamento della ridondanza è l'\textbf{Immediate Refresh}: essa viene aggiornata ad ogni inserimento/modifica/cancellazione nella relazione \textbf{Recensione}.
\\ \\
Di seguito è riportata la frequenza giornaliera delle attività che coinvolgono tale ridondanza: \\ \\
\begin{tabular}{|C{2.2cm}|C{3.8cm}|C{4.4cm}|}
\hline
    \textbf{Azione} & \textbf{FrequenzaGiornaliera} & \textbf{Nota} \\
\hline
Film Migliori & 30 000 000 & Si considera che ogniqualvolta un utente si connetta gli venga mostrato il film con recensioni migliori come raccomandazione automatica \\
\hline
Inserimento Recensione & 70 & Si stima che le 25 000 recensioni siano state accumulate nel corso di un anno \\
\hline
\end{tabular}
\\ \\ 
\textbf{Input}: Nulla \\
\textbf{Output}: Titoli e media recensioni dei 20 Film con media più alta
\\ \\
I Volumi interessati sono: \\ \\ 
\begin{tabular}{|C{2.6cm}|C{2.6cm}|C{2.6cm}|}
\hline
    \textbf{Nome} & \textbf{Tipo} & \textbf{Volume} \\
\hline
 Film & E & 1 000 \\
\hline
 Recensione & R & 25 000 \\
\hline
\end{tabular}
\\ \\  
%
%
%
% Fine Pagina 35
%
% ~ ~ ~ ~ ~ ~ ~ ~ ~ ~ ~
%
% Inizio Pagina 36
%
%
%
Gli accessi per Visualizzazione sarebbero in media, senza ridondanza: \\ \\
\begin{tabular}{|C{1.7cm}|C{1.7cm}|C{1.3cm}|C{0.8cm}|C{5.1cm}|}
\hline
    \textbf{Concetto} & \textbf{Costrutto} & \textbf{Accessi} & \textbf{Tipo} & \textbf{Note} \\
\hline
    Recensione & R & 25 000 & R & Devono essere lette tutte le occorrenze, raggruppate in base alla chiave di \textbf{Film} e ordinate in base alla media di \textit{Voto} \\
\hline
    Film & E & 20 & R & Noti gli \textit{ID} dei \textbf{Film} di cui si deve riportare il \textit{Titolo} si accede direttamente a essi dalla chiave \\
\hline
    \multicolumn{2}{|c|}{\textbf{Totale:}} & 
    \multicolumn{2}{c|}{25 020} & 
    Soluzione non ottimale \\ 
\hline
\end{tabular} \\ \\ \\ 
Mentre, con la ridondanza sono in media: \\ \\
\begin{tabular}{|C{1.7cm}|C{1.7cm}|C{1.3cm}|C{0.8cm}|C{5.1cm}|}
\hline
    \textbf{Concetto} & \textbf{Costrutto} & \textbf{Accessi} & \textbf{Tipo} & \textbf{Note} \\
\hline
    Film & E & 1 000 & R & Si leggono tutti i Film e si trovano le migliori 20 medie \\
\hline
    \multicolumn{2}{|c|}{\textbf{Totale:}} & 
    \multicolumn{2}{c|}{1 000} & 
    Soluzione ottimale \\ 
\hline
\end{tabular} \\ \\ \\
Gli accessi relativi all'aggiornamento della ridondanza sono invece:\\ \\
\begin{tabular}{|C{1.7cm}|C{1.7cm}|C{1.3cm}|C{0.8cm}|C{5.1cm}|}
\hline
    \textbf{Concetto} & \textbf{Costrutto} & \textbf{Accessi} & \textbf{Tipo} & \textbf{Note} \\
\hline
    Film & E & 1 & W & Conoscendo la chiave di \textbf{Film} e memorizzando sia \textit{NumeroRecensioni} che \textit{MediaRecensioni} si possono calcolare i nuovi valori senza effettuare altre letture \\
\hline
    \multicolumn{2}{|c|}{\textbf{Totale:}} & 
    \multicolumn{2}{c|}{1} & \\ 
\hline
\end{tabular} \\ \\ \\
Gli accessi giornalieri delle due tipologie sono quindi: \\ \\
\begin{tabular}{|C{3cm}|C{2.5cm}|C{2.5cm}|C{2cm}|}
\cline{2-4}
   \multicolumn{1}{c}{}  & \multicolumn{3}{|c|}{Accessi al giorno} \\
\hline
    \textbf{Metodologia} & \textbf{Lettura} & \textbf{Scrittura} & \textbf{Totale} \\
\hline
    Senza ridondanza & 25 X $ 10^{9} $ & 0 & 25 X $ 10^{9} $ \\
\hline
    Con ridondanza & 1 X $ 10^{9} $ & 70 & 1 X $ 10^{9} $\\
\hline
\end{tabular}



%
%
%
% Fine Pagina 36
%
% ~ ~ ~ ~ ~ ~ ~ ~ ~ ~ ~
%
% Inizio Pagina 37
%
%
%
\subsection{Analisi delle Operazioni}
Le stime sulle tabelle degli accessi vengono fatte supponendo di non sapere assolutamente nulla sulla struttura dati utilizzati per implementare sia le Relazioni che le Entità, assumiamo solamente che essendo a conoscenza, per interezza, dell'identificativo di un costrutto si possa accedere ad esso in maniera puntuale. \\
Pertanto, non prenderemo in considerazione, in quanto logicamente sbagliato non trattandosi né di progettazione logica né di progettazione fisica, l'eventualità di ottimizzazioni di accessi quali la presenza di indici oppure Foreign Key. \\
Le operazioni di scrittura nel conteggio finale verranno raddoppiate in quanto evidentemente più pesanti di una lettura.
\subsubsection{Vincite di un Film}
\textbf{Descrizione}: Non appena viene selezionato il film da visionare la piattaforma carica una lista contenenti sia i premi vinti dal film, compredenti anche le date, e il numero di premi totali vinti. \\ \\ 
\textbf{Input}: ID di Film\\
\textbf{Output}: Lista di Macrotipo, Microtipo e Data, Numero di Premi \\ 
\textbf{Frequenza Giornaliera}: 500 000 Poiché un \textbf{Utente}, in media, vede un \textbf{Film} ogni due giorni e i premi vengono mostrati nella pagina dell'app relativa al \textbf{Film} \\ \\
\textbf{Porzione di Diagramma Interessato}: \\ 
\begin{center}
    \centering
    \includegraphics[width=0.55\linewidth]{assets/Operazione_1.png}
\end{center}
\textbf{Porzione della Tavola dei Volumi}: \\ \\
\begin{tabular}{|C{2.6cm}|C{2.6cm}|C{2.6cm}|}
\hline
    \textbf{Concetto} & \textbf{Tipo} & \textbf{Volume} \\
\hline
 Premio & E & 50 \\
\hline
 Film & E & 1 000 \\
\hline
 Vincita & E & 1 500 \\
\hline
 VincitaFilm & R & 1 500 \\
\hline
 VincitaPremio & R & 1 500 \\
\hline
\end{tabular} \\ \\ \\ 
\textbf{Tavola degli Accessi}: \\ \\
%
%
%
% Fine Pagina 38
%
% ~ ~ ~ ~ ~ ~ ~ ~ ~ ~ ~
%
% Inizio Pagina 38
%
%
%
\begin{tabular}{|C{1.7cm}|C{1.7cm}|C{1.3cm}|C{0.8cm}|C{5.1cm}|}
\hline
    \textbf{Concetto} & \textbf{Costrutto} & \textbf{Accessi} & \textbf{Tipo} & \textbf{Note} \\
\hline
    VincitaFilm & R & 1 500 & R & Conoscendo solo la chiave di \textbf{Film} si deve leggere necessariamente tutto \textbf{VincitaFilm} per trovare la chiave di \textbf{Vincita} contenente tutte le informazioni richieste.\\
\hline
    \multicolumn{2}{|c|}{\textbf{Totale:}} & \multicolumn{2}{c|}{1 500} & \\ 
\hline
    \multicolumn{2}{|c|}{\textbf{Totale Giornaliero:}} & \multicolumn{2}{c|}{750 000 000} & \\ 
\hline
\end{tabular} 
\newpage
\subsubsection{Generi di un Film}
\textbf{Descrizione}: Non appena viene selezionato un \textbf{Film} viene caricata una lista contenente i \textbf{Generi} a cui appartiene il \textbf{Film}. Oltretutto, controllando il tipo di abbonamento dell'\textbf{Utente} si verifica che egli sia abilitato o meno alla fruizione del contenuto.\\ \\ 
\textbf{Input}: ID, Codice \\ 
\textbf{Output}: Lista Generi, Abiltazione alla fruizione \\ 
\textbf{Frequenza Giornaliera}: 500 000 un \textbf{Utente}, in media, vede un \textbf{Film} ogni due giorni \\ \\
\textbf{Porzione di Diagramma Interessato}:
\begin{center}
\centering
\includegraphics[width=0.8\linewidth]{assets/Operazione_2 (2).png}
\end{center} 
%
%
%
% Fine Pagina 38
%
% ~ ~ ~ ~ ~ ~ ~ ~ ~ ~ ~
%
% Inizio Pagina 39
%
%
%
\textbf{Porzione della Tavola dei Volumi}: \\ \\
\begin{tabular}{|C{2.6cm}|C{2.6cm}|C{2.6cm}|}
\hline
    \textbf{Concetto} & \textbf{Tipo} & \textbf{Volume} \\
\hline
Genere & E & 27 \\
\hline
GenereFilm & R & 2000 \\ 
\hline
Film & E & 1000 \\
\hline
Utente & E & 1 000 000 \\
\hline
Abbonamento Utente & R & 1 000 000 \\
\hline
Abbonamento & E & 5 \\
\hline
Esclusione & R & 15 \\
\hline
\end{tabular} \\
\textbf{Tavola degli Accessi}: \\ \\
\begin{tabular}{|C{1.7cm}|C{1.7cm}|C{1.3cm}|C{0.8cm}|C{5.1cm}|}
\hline
    \textbf{Concetto} & \textbf{Costrutto} & \textbf{Accessi} & \textbf{Tipo} & \textbf{Note} \\
\hline
    GenereFilm & R & 2 000 & R & Conoscendo solo la chiave di \textbf{Film} si devono leggere tutte le occorrenze di \textbf{GenereFilm}. Si stima che i generi ottenuti siano due poiché un film ha in media due generi.\\
\hline
    Esclusione & R & 15 & R &  Conoscendo solo la chiave di \textbf{Genere} si devono leggere tutte le occorrenze di \textbf{Esclusione}. Si stime che che gli abbonamenti ottenuti siano uno poiché un genere ogni due è escluso in media in un abbonamento \\
\hline
    Abbona- mento Utente & R & 1 & R & Conoscendo sia la chiave di \textbf{Utente} che la chiave di \textbf{Utente} si può accedere direttamente a \textbf{AbbonamentoUtente} \\
\hline

    \multicolumn{2}{|c|}{\textbf{Totale:}} & \multicolumn{2}{c|}{2 016} &  \\ 
\hline
    \multicolumn{2}{|c|}{\textbf{Totale Giornaliero:}} & \multicolumn{2}{c|}{1 008 000 000} & \\ 
\hline
\end{tabular} \\ \\ \\ \\ \\ \\ 
\subsubsection{Massima Risoluzione di un Film}
\textbf{Descrizione}: dato un \textbf{Film} restituire, tra tutte le \textbf{Edizioni}, il/i \textbf{File} con la \textit{Risoluzione} maggiore permessa all'utente. \\ \\
\textbf{Input}: ID di Film, ID di Utente \\
\textbf{Output}: ID di File con associata Risoluzione \\
\textbf{Frequenza Giornaliera}: 50 000 Perché si stima che ogni \textbf{Utente} guardi un \textbf{Film} ogni due giorni e che una volta su 10 cerchi il \textbf{File} con la massima \textit{Risoluzione} possibile \newpage
\textbf{Porzione di Diagramma interessato}: \\ \\
\begin{center}
    \centering
    \includegraphics[width=\linewidth]{assets/Operazione_3.png}
\end{center}
%
%
%
% Fine Pagina 39
%
% ~ ~ ~ ~ ~ ~ ~ ~ ~ ~ ~
%
% Inizio Pagina 40
%
%
%
\textbf{Porzione della Tabella dei Volumi}: \\ \\
\begin{tabular}{|C{2.6cm}|C{2.6cm}|C{2.6cm}|}
\hline
    \textbf{Concetto} & \textbf{Tipo} & \textbf{Volume} \\
\hline
 EdizioneFilm & R & 1 500 \\
\hline
 EdizioneFile & R & 4 500 \\
\hline
 VideoFile & R & 4 500 \\
\hline
 Utente & E & 1 000 000 \\
\hline 
 Abbonamento & E & 5 \\
\hline 
 Abbonamento- Utente & R & 1 000 000 \\
\hline
\end{tabular} \\ \\ \\
\textbf{Tavola degli Accessi}: \\ \\ 
\begin{tabular}{|C{1.7cm}|C{1.7cm}|C{1.3cm}|C{0.8cm}|C{5.1cm}|}
\hline
    \textbf{Concetto} & \textbf{Costrutto} & \textbf{Accessi} & \textbf{Tipo} & \textbf{Note} \\
\hline
    Abbona- mento & E & 5 & R & Occorre leggere la massima risoluzione consentita per ogni abbonamento \\
\hline
    Abbona- mento- Utente & R & 5 & R & Avendo sia la chiave di \textbf{Utente} che la chiave di \textbf{Abbonamento} si accede puntualmente per vedere quale abbonamento ha un'occorrenza effettiva. \\
\hline
    Edizione-Film & R & 1 500 & R & Conoscendo solo la chiave di \textbf{Film} si accede a tutta la relazione per trovare l'ID dell'\textbf{Edizione} associata. \\
\hline
\end{tabular}
\begin{tabular}{|C{1.7cm}|C{1.7cm}|C{1.3cm}|C{0.8cm}|C{5.1cm}|}
\hline
    File-Edizione & E & 4 500 & R &  Conoscendo solo la chiave di \textbf{Edizione} si accede a tutta la relazione per trovare l'ID dell'\textbf{File} associato. \\
\hline
    VideoFile & R & 4 500 & R & Noti gli ID dei \textbf{File} interessati si estrae la \textit{Risoluzione}  trovandone la massima \\
\hline
    \multicolumn{2}{|c|}{\textbf{Totale:}} & \multicolumn{2}{c|}{10 510} &  \\ 
\hline
    \multicolumn{2}{|c|}{\textbf{Totale Giornaliero:}} & \multicolumn{2}{c|}{525 500 000} & \\ 
\hline
\end{tabular} \\ \\ \\
\subsubsection{Numero di Film non disponibili in un Abbonamento}
\textbf{Descrizione}: Quando l'utente sceglie quale abbonamento acquistare può informarsi sul numero di \textbf{Film} esclusi da uno specifico \textbf{Abbonamento} \\ \\
\textbf{Input}: Tipo \\
\textbf{Output}: Numero di Film Esclusi \\
\textbf{Frequenza Giornaliera}: 80 000 Poiché gli abbonamenti hanno durata mensile si stima che ogni fine mese il 50\% degli utenti decida di cambiare abbonamento e, dunque, controlli il numero di \textbf{Film} esclusi per tutti i 5 abbonamenti \\ \\
\textbf{Porzione di Diagramma Interessato}: \\ \\
%
%
%
% Fine Pagina 40
%
% ~ ~ ~ ~ ~ ~ ~ ~ ~ ~ ~
%
% Inizio Pagina 41
%
%
%
\begin{center}
\centering 
\includegraphics[width=\linewidth]{assets/Operazione_4.png}
\end{center} \newpage
\textbf{Tavola dei Volumi}: \\ \\
\begin{tabular}{|C{2.6cm}|C{2.6cm}|C{2.6cm}|}
\hline
    \textbf{Concetto} & \textbf{Tipo} & \textbf{Volume} \\
\hline
 Abbonamento & E & 5 \\
\hline
 Esclusione & R & 15 \\
\hline
 Genere & E & 27 \\
\hline
 GenereFilm & R & 2 000 \\
\hline
 Film & E & 1 000 \\
\hline
 EdizioneFilm & R & 1 500 \\
\hline 
 Edizione & E & 1 500 \\
\hline 
 EdizioneFile & R & 4 500 \\ 
\hline
 File & E & 4 500 \\ 
\hline
 VideoFile & R & 4 500 \\ 
\hline 
 FormatoCodifica & E & 100 \\ 
\hline 
\end{tabular} \\ \\ \\
\textbf{Tavola degli Accessi}: \\ \\
\begin{tabular}{|C{1.7cm}|C{1.7cm}|C{1.3cm}|C{0.8cm}|C{5.1cm}|}
\hline
    \textbf{Concetto} & \textbf{Costrutto} & \textbf{Accessi} & \textbf{Tipo} & \textbf{Note} \\
\hline
    Esclusione & R & 15 & R & Conoscendo la chiave di \textbf{Abbonamento} si accede a tutta  \textbf{Esclusione}. Si assume che ogni abbonamento esclude, in media, 3 generi. \\
\hline
    GenereFilm & R & 2 000 & R & Conoscendo solo la chiave di \textbf{Genere} si accede direttamente a tutta \textbf{GenereFilm}. \\
\hline 
    Abbona- mento & E & 1 & R & Si legge la massima risoluzione disponibile \\
\hline
    VideoFile & R & 4 500 & R & Si legge la risoluzione di tutti i \textbf{File}. \\
\hline
    Edizione-File & R & 4 500 & R & Si legge la minima risoluzione per ogni \textbf{Edizione}. \\
\hline
    Film-Edizione & R & 1 500 & R & Si legge la minima risoluzione per ogni \textbf{Film}. \\
\hline
    \multicolumn{2}{|c|}{\textbf{Totale:}} & \multicolumn{2}{c|}{12 516} & \\
\hline
    \multicolumn{2}{|c|}{\textbf{Totale Giornaliero:}} & \multicolumn{2}{c|}{1 001 280 000} & \\ 
\hline
\end{tabular} \newpage
%
%
%
% Fine Pagina 41
%
% ~ ~ ~ ~ ~ ~ ~ ~ ~ ~ ~
%
% Inizio Pagina 42
%
%
%
\subsubsection{Film disponibili in Lingua specifica}
\textbf{Descrizione}: data una \textbf{Lingua} restituire, tutti i titoli di \textbf{Film} che hanno almeno un \textbf{Doppiaggio}, presente nel sistema, nella lingua richiesta. \\ \\
\textbf{Input}: ID di Lingua \\
\textbf{Output}: Lista di ID e Titoli di Film che soddisfano il criterio \\
\textbf{Frequenza Giornaliera}: 100 000 Perché si stima che ogni Utente guardi un Film ogni due giorni e che un quinto delle volte ne scelga uno in base ad una Lingua specifica. La ricerca dei Film disponibili in una lingua avviene quindi, per ogni Utente, una volta ogni 10 giorni. \\ \\
\textbf{Porzione di Diagramma interessato}: \\ 
\begin{center}
    \centering
    \includegraphics[width=0.9\linewidth]{assets/Operazione_5.png}
\end{center}
\textbf{Porzione della Tabella dei Volumi}: \\ \\ 
\begin{tabular}{|C{2.6cm}|C{2.6cm}|C{2.6cm}|}
\hline
    \textbf{Concetto} & \textbf{Tipo} & \textbf{Volume} \\
\hline
 Lingua & E & 141 \\
\hline
 Doppiaggio & R & 18 000 \\
\hline
 EdizioneFile & R & 4 500 \\
\hline
 EdizioneFilm & R & 1 500 \\
\hline
 Film & E & 1 000 \\
\hline
 File & E & 4 500 \\
\hline 
\end{tabular} \\ \\ \\
\textbf{Tavola degli Accessi}: \\ \\ 
\begin{tabular}{|C{1.7cm}|C{1.7cm}|C{1.3cm}|C{0.8cm}|C{5.1cm}|}
\hline
    \textbf{Concetto} & \textbf{Costrutto} & \textbf{Accessi} & \textbf{Tipo} & \textbf{Note} \\
\hline
    Doppiaggio & R & 18 000 & R & Conoscendo solo la chiave di \textbf{Lingua} va letta tutta la relazione. Si stimano circa 130 file trovati (18 000 /\ 141) \\
\hline
    Edizione-File & R & 4 500 & R & Conoscendo solo la chiave di \textbf{File} va letta tutta la relazione. Si stimano circa 43 edizioni trovate (ci sono 3 file per ogni edizione) \\
\hline
    Edizione-Film & R & 1 500 & R & Conoscendo solo la chiave di \textbf{Edizione} va letta tutta la relazione. Si stimano circa 29 film (un film ogni due ha due edizioni) \\
\hline
    Film & E & 29 & R & Noti gli ID dei \textbf{Film} cercati se ne estraggono i \textit{Titoli} \\
\hline
    \multicolumn{2}{|c|}{\textbf{Totale:}} & 
    \multicolumn{2}{c|}{24 029} & \\ 
\hline
    \multicolumn{2}{|c|}{\textbf{Totale Giornaliero:}} & 
    \multicolumn{2}{c|}{2 402 500 000} & \\ 
\hline
\end{tabular} \newpage
\subsubsection{Film poco popolari}
\textbf{Descrizione}: Un utente potrebbe voler conoscere tutti i film, non esclusi dal suo abbonamento, nei quali tutti gli attori ed anche il regista hanno una popolarità al di sotto di una certa soglia .\\ \\
\textbf{Input}: Popolarità, Codice Utente \\
\textbf{Output}: Titoli di Film disponibili \\
\textbf{Frequenza Giornaliera}: 10 000 Si stima che il 1\% delle volte che un utente si connetta vada a controllare la lista dei \textbf{Film} poco popolari \\ \\
\textbf{Porzione di Diagramma Interessato}: \\
\begin{center}
    \centering
    \includegraphics[width=0.85\linewidth]{assets/operazione_6.png}
\end{center}
\textbf{Porzione della Tabella dei Volumi}: \\ \\
\begin{tabular}{|C{2.6cm}|C{2.6cm}|C{2.6cm}|}
\hline
    \textbf{Concetto} & \textbf{Tipo} & \textbf{Volume} \\
\hline 
    Artista & E & 4 000 \\ 
\hline
    Recitazione & R & 20 000 \\ 
\hline
    Direzione & R & 1 000\\ 
\hline
    Film & E & 1 000 \\ 
\hline
    Utente & E & 1 000 000 \\ 
\hline
    Abbonamento- Utente & R & 1 000 000\\ 
\hline
    Abbonamento & E & 5 \\ 
\hline
    Esclusione & R & 15 \\ 
\hline
    Genere & E & 27\\ 
\hline
    GenereFilm & R & 2 000 \\ 
\hline
\end{tabular}
\\ \\ \\ \\ \\ 
\textbf{Tavola degli Accessi}: \\ \\
\begin{tabular}{|C{1.7cm}|C{1.7cm}|C{1.3cm}|C{0.8cm}|C{5.1cm}|}
\hline
    \textbf{Concetto} & \textbf{Costrutto} & \textbf{Accessi} & \textbf{Tipo} & \textbf{Note} \\
\hline
    Abbona- mento & E & 5 & R & Si leggono tutti gli \textbf{Abbonamenti} \\ 
\hline
    Abbon- Utente & R & 5 & R & Viene letto l'\textit{ID} dell'attuale \textbf{Abbonamento} provando ad accedere con la chiave di \textbf{Utente} e le 5 chiavi di \textbf{Abbonamento} \\
\hline
    Esclusione & R & 15 & R & Vengono letti tutti i \textbf{Generi} esclusi accedendo solo con la chiave di \textbf{Abbonamento}. \\
\hline
    GenereFilm & R & 2 000 & R & Vengono letti i \textbf{Film} appartenenti ai generi esclusi \\
\hline 
    Regia & R & 1000 & R & Vengono letti i \textbf{Registi} di tutti i film \\
\hline 
    Recitazione & R & 20 000 & R & Vengono letti gli \textbf{Attori} di tutti i film \\
\hline
    Artista & E & 4 000 & R & Vengono lette le \textbf{Popolarità} di tutti gli Artisti \\
\hline
\multicolumn{2}{|c|}{\textbf{Totale:}} & 
    \multicolumn{2}{c|}{27 025} & \\ 
\hline
    \multicolumn{2}{|c|}{\textbf{Totale Giornaliero:}} & 
    \multicolumn{2}{c|}{2 702 500 000} & \\ 
\hline
\end{tabular} \newpage
%
%
%
% Fine Pagina 43
%
% ~ ~ ~ ~ ~ ~ ~ ~ ~ ~ ~
%
% Inizio Pagina 44
%
%
%




\subsubsection{Cambio Abbonamento}
\textbf{Descrizione}: Un \textbf{Utente} può decidere di cambiare piano di abbonamento, ovviamente prima di fare ciò occorre controllare che sia in pari coi pagamenti. \\ \\
\textbf{Input}: Codice, Tipo \\
\textbf{Output}: \\
\textbf{Frequenza Giornaliera}: 1 667 Poiché gli abbonamenti hanno durata mensile si stima che ogni fine mese il 5\% degli utenti decida di cambiare abbonamento \\ \\
\textbf{Porzione di Diagramma Interessata}: \\
%
%
%
% Fine Pagina 44
%
% ~ ~ ~ ~ ~ ~ ~ ~ ~ ~ ~
%
% Inizio Pagina 45
%zzzz
%
%
\begin{center}
\centering
\includegraphics[width=0.8\linewidth]{assets/Operazione_7.png}
\end{center} 
\textbf{Porzione della Tabella dei Volumi}: \\ \\
\begin{tabular}{|C{2.6cm}|C{2.6cm}|C{2.6cm}|}
\hline
    \textbf{Concetto} & \textbf{Tipo} & \textbf{Volume} \\
\hline
 Utente & E & 1 000 000 \\
\hline
 Abbonamento & E & 5 \\
\hline
 Fattura & E & 12 000 000 \\
\hline
 Pagamento & E & 11 000 000 \\
\hline
 Abbonamento- Utente & R & 1 000 000 \\
\hline
 Emissione & R & 12 000 000 \\
\hline
 Pagamento- Fattura & R & 11 000 000 \\
\hline
\end{tabular} \\ \\ \\ \\ 
\textbf{Tavola degli Accessi}: \\ \\
\begin{tabular}{|C{1.7cm}|C{1.7cm}|C{1.3cm}|C{0.8cm}|C{5.1cm}|}
\hline
    \textbf{Concetto} & \textbf{Costrutto} & \textbf{Accessi} & \textbf{Tipo} & \textbf{Note} \\
\hline
    Abbona- mento & R & 5 & R & Vengono letti i prezzi degli \textbf{Abbonamenti}\\
\hline
    Abbon- Utente & R & 5 & R & Si determina l'\textit{ID} dell'attuale \textbf{Abbonamento} provando ad accedere con le 5 chiavi di \textbf{Abbonamento}. Se l'\textit{ID} ottenuto è il medesimo al quale si vuole passare (caso improbabile) si termina l'operazione \\
\hline
    Emissione & R & 12 000 000 & R & Dall'\textit{ID} dell'\textbf{Utente} si leggono le \textbf{Emissioni} relitive a egli. Si ottengono quindi gli \textit{ID} delle \textbf{Fatture}. Si ricorda che solo i dati degli ultimi 12 mesi sono memorizzati nella Base di Dati \\
\hline
    Pagamento & E & 11 000 000 & R & Avendo \textbf{Pagamento} stessa chiave di \textbf{Fattura}, ne si controlla l'esistenza (tentativo di lettura) \\
\hline
    Abbon- Utente & R & 1 & W & Si modifica la relazione sostituendo il vecchio \textbf{Abbonamento} con quello nuovo \\
\hline
    \multicolumn{2}{|c|}{\textbf{Totale:}} & \multicolumn{2}{c|}{23 000 012} & \\ 
\hline
    \multicolumn{2}{|c|}{\textbf{Totale Giornaliero:}} & \multicolumn{2}{c|}{38 341 020 004} & \\ 
\hline
\end{tabular} \newpage
\subsubsection{Film con Recensioni migliori}
\textbf{Descrizione}: Restituire i migliori 20 \textbf{Film} con la più alta media di \textit{Voto} nelle \textbf{Recensioni} associate. \\ \\
\textbf{Input}: Nulla \\
\textbf{Output}: Lista di \textit{ID} e \textit{Titoli} di \textbf{Film}. \\
\textbf{Frequenza Giornaliera}: 500 000 Si considera che ogni volta che un utente si connette gli vengano mostrato i \textbf{Film} con \textbf{Recensioni} migliori come raccomandazione automatica \\ \\ 
\textbf{Porzione di Diagramma interessato}: \\ \\
\begin{center}
\centering
\includegraphics[width=0.35\linewidth]{assets/Operazione_8.png}
\end{center}
%
%
%
% Fine Pagina 46
%
% ~ ~ ~ ~ ~ ~ ~ ~ ~ ~ ~
%
% Inizio Pagina 47
%
%
%
\textbf{Porzione della Tabella dei Volumi}: \\ \\
\begin{tabular}{|C{2.6cm}|C{2.6cm}|C{2.6cm}|}
\hline
    \textbf{Concetto} & \textbf{Tipo} & \textbf{Volume} \\
\hline
 Film & E & 1 000 \\
\hline
 Recensione & R & 25 000 \\
\hline
\end{tabular} \\ \\ \\ \\ \\ \\ \\ \\ 
% Seconda Tabella degli Accessi senza Rid
\textbf{Tavola degli Accessi}: \\ \\
\begin{tabular}{|C{1.7cm}|C{1.7cm}|C{1.3cm}|C{0.8cm}|C{5.1cm}|}
\hline
    \textbf{Concetto} & \textbf{Costrutto} & \textbf{Accessi} & \textbf{Tipo} & \textbf{Note} \\
\hline
    Film & E & 1 000 & R & La ridondanza \textit{MediaRecensioni} permettere di individuare il miglior Film senza dover calcolare le recensioni medie \\
\hline
    \multicolumn{2}{|c|}{\textbf{Totale:}} & \multicolumn{2}{c|}{1 000} &\\ 
\hline
    \multicolumn{2}{|c|}{\textbf{Totale Giornaliero:}} & \multicolumn{2}{c|}{500 000 000} & Soluzione Ottimale \\ 
\hline
\end{tabular} \\ \\ \\ \\ 
\textbf{Tavola degli Accessi}: \\ \\
\begin{tabular}{|C{1.7cm}|C{1.7cm}|C{1.3cm}|C{0.8cm}|C{5.1cm}|}
\hline
    \textbf{Concetto} & \textbf{Costrutto} & \textbf{Accessi} & \textbf{Tipo} & \textbf{Note} \\
\hline
    Recensione & R & 25 000 & R & Bisogna leggere tutte le \textbf{Recensioni} per calcolare la media per film \\
\hline
& & & & \\
    Film & E & 1 000 & R & Occorre leggere i Titoli dei \textbf{Film} \\
& & & & \\
\hline
    \multicolumn{2}{|c|}{\textbf{Totale:}} & \multicolumn{2}{c|}{26 000} & \\ 
\hline
    \multicolumn{2}{|c|}{\textbf{Totale Giornaliero:}} & \multicolumn{2}{c|}{98 000 000 000} & Soluzione Non Ottimale \\ 
\hline
\end{tabular} \\ \\
\section{Progettazione Logica}
\subsection{Ridenominazione Nomi}
\subsection{Traduzione in schema Logico}
%
% Si potrebbe pensare di fare una lista come quella di Tau e Trains e poi agiungere una tabella
% in cui vi sono: ConcettiER, Traduzione(senza attributi) e Note
%
%
\begin{tabular}{|C{2.7cm}|C{8cm}|C{0.8cm}|}
\hline
& & \\
    \textbf{Concetti ER} & \textbf{Traduzione} & \textbf{Note} \\
& & \\
\hline
& & \\
    Film, Produzione, Regia & Film(\underline{ID}, Titolo, Descrizione, Anno, CasaProduzione, NomeRegista, CognomeRegista, MediaRecensioni, NumeroRecensioni) & \\
& & \\
\hline
& & \\
    Artista & Artista(\underline{Nome}, \underline{Cognome}, Popolarità) & \\
& & \\
\hline
& & \\
    Premio, VincitaPremio, Vincita, VincitaArtista, VincitaFilm & VincitaPremio(\underline{Macrotipo}, \underline{Microtipo}, \underline{Data}, Film, NomeArtista, CognomeArtista) & 1 \\
& & \\
\hline
& & \\
    Recitazione & Recitazione(\underline{Film}, \underline{NomeAttore}, \underline{CognomeAttore}) & \\
& & \\
\hline
& & \\
    Critico & Critico(\underline{Codice}, Nome, Cognome) & \\
& & \\
\hline
& & \\
    Critica & Critica(\underline{Critico}, \underline{Film}, Testo, Data, Voto) & \\
& & \\
\hline
& & \\
    Recensione & Recensione(\underline{Utente}, \underline{Film}, Voto) & \\
& & \\
\hline 
\end{tabular} \\ \\


\begin{tabular}{|C{2.7cm}|C{8cm}|C{0.8cm}|}
\hline
& & \\
    \textbf{Concetti ER} & \textbf{Traduzione} & \textbf{Note} \\
& & \\
\hline
& & \\
    Genere & Genere(\underline{Nome}) & \\
& & \\
\hline
& & \\
    GenereFilm & GenereFilm(\underline{Genere}, \underline{Film}) & \\
& & \\
\hline
& & \\
    CasaProduzione, Appartenenza & CasaProduzione(\underline{Nome}, Paese) & \\ 
& & \\
\hline 
& & \\
    Lingua & Lingua(\underline{Nome}) & \\ 
& & \\
\hline 
& & \\
    Doppiaggio & Doppiaggio(\underline{File}, \underline{Lingua}) & \\ 
& & \\
\hline 
& & \\
    Sottotitolaggio & Sottotitolaggio(\underline{File}, \underline{Lingua}) & \\
& & \\
\hline
& & \\
    Paese & Paese(\underline{Codice}, Nome, Posizione) & \\
& & \\
\hline
& & \\
    File, VideoFile, AudioFile, EdizioneFile & File(\underline{ID}, Dimensione, BitRate, FormatoContenitore, FamigliaAudio, VersioneAudio, FamigliaVideo, VersioneVideo, Risoluzione, FPS, BitDepth, Frequenza, Edizione) & \\
& & \\
\hline 
& & \\
    FormatoCodifica & FormatoCodifica(\underline{Famiglia}, \underline{Versione}, Lossy, MaxBitrate) & \\
& & \\
\hline 
& & \\
    Edizione, EdizioneFilm & Edizione(\underline{ID}, Film, Anno, Tipo, Lunghezza, Rapporto D'Aspetto) & \\ 
& & \\
\hline 
\end{tabular} \\ \\
%
%
%
% Fine Pagina 48
%
% ~ ~ ~ ~ ~ ~ ~ ~ ~ ~ ~ ~ 
%
% Inizio Pagina 49
%
%
%
\begin{tabular}{|C{2.7cm}|C{8cm}|C{0.8cm}|}
\hline
& & \\
    \textbf{Concetti ER} & \textbf{Traduzione} & \textbf{Note} \\
& & \\
\hline 
& & \\
    
    Restrizione & Restrizione(\underline{Edizione}, \underline{Paese}) & \\
& & \\
\hline
& & \\
    IP Range, Geolocal & IPRange(\underline{Inizio}, \underline{Fine}, \underline{DataInizio}, DataFine, Paese) & \\ 
& & \\ 
\hline
& & \\
    Server & Server(\underline{ID}, LunghezzaBanda, MTU, MaxConnessioni, CaricoAttuale, Posizione) & \\
& & \\
\hline
& & \\
    P.o.P. & P.o.P.(\underline{File}, \underline{Server}) & \\ 
& & \\ 
\hline
& & \\
    Distanza- Precalcolata & DistanzaPrecalcolata(\underline{Paese}, \underline{Server}, ValoreDistanza) & \\ 
& & \\ 
\hline
& & \\
    Visualizzazione, VisualizzazioneEdizione, VisualizzazioneConnession & Visualizzazione(\underline{Timestamp}, \underline{Edizione}, \underline{Utente}, \underline{IP}, \underline{InizioConnessione}) & \\
& & \\
\hline
& & \\
    Erogazione & Erogazione(\underline{Timestamp}, \underline{Edizione}, \underline{Utente}, \underline{IP}, \underline{InizioConnessione}, InizioErogazione, Server) & 2 \\
& & \\
\hline
& & \\
    Connessione, ConnessioneUtente & Connessione(\underline{Utente}, \underline{IP}, \underline{Inizio}, Fine, Hardware) & \\
& & \\
\hline
& & \\
    Utente, AbbonamentoUtente & Utente(\underline{Codice}, Nome, Cognome, Email, Password, Abbonamento, DataInizioAbbonamento) & \\
& & \\
\hline
\end{tabular} \\ \\

\begin{tabular}{|C{2.7cm}|C{8cm}|C{0.8cm}|}
\hline
& & \\
    \textbf{Concetti ER} & \textbf{Traduzione} & \textbf{Note} \\
& & \\
\hline 
& & \\
    Abbonamento & Abbonamento(\underline{Tipo}, Tariffa, Durata, Definizione, Offline, MaxOre, GBMensili) & \\
& & \\
\hline
& & \\
    Esclusione & Esclusione(\underline{Abbonamento}, \underline{Genere}) & \\
& & \\
\hline
& & \\
    Fattura, Emissione, Pagamento, PagamentoCarta & Fattura(\underline{ID}, Utente, DataEmissione, DataPagamento, CartaDiCredito) & 3\\
& & \\
\hline
& & \\
    Carta di Credito & CartaDiCredito(\underline{PAN}, Scadenza, CVV) & \\
& & \\
\hline
& & \\
    Film Vis Giorni, Paese Vis Giorni, Visualizzazioni Giornaliere & VisualizzazioniGiornaliere(\underline{Film}, \underline{Data}, \underline{Paese}, NumeroVisualizzazioni) & \\
& & \\
\hline
\end{tabular} \\ \\



Note: \\ 
\begin{enumerate}
    \item Stimiamo che la metà dei premi siano vinti da un artista, dunque è più conveniente avere due valori nulli invece di ripetere metà delle volte i 3 attributi chiave in una tabella che avrebbe tradotto la relazione VincitaArtista
    
    \item Le Erogazioni conterranno i dati solo delle visualizzazioni attive. Se \textbf{Erogazione} fosse stato tradotto come un attributo \textit{ServerAttuale} in \textbf{Visualizzazione}, al termine di essa sarebbe dovuto essere impostato come nullo. Visualizzazione avrebbe avuto quindi la maggior parte delle occorrenze con un campo nullo.
    
    \item Pagamento e PagamentoCarta confluiscono dentro Fattura in quanto entrambe le entità condividono la chiave e si stima che al più 1/12 delle fatture possa non essere pagata. Si ha quindi anche un risparmio di memoria in quanto non serve memorizzare più volte la chiave di Fattura
\end{enumerate}


\subsection{Vincoli}
\subsubsection{Vincoli di Integrità Referenziale}
\begin{enumerate}
\itemsep-0.25em % Riduce la distanza tra gli item
    \item \textit{CasaProduzione} di \textbf{Film} in \textit{Nome} di \textbf{CasaProduzione}
    
    \item \textit{NomeRegista}, \textit{CognomeRegista} di \textbf{Film} in \textit{Nome}, \textit{Cognome} di \textbf{Artista}
    
    \item \textit{Film} di \textbf{VincitaPremio} in \textit{ID} di \textbf{Film}
    
    \item \textit{NomeArtista}, \textit{CognomeArtista} di \textbf{VincitaPremio} in \textit{Nome}, \textit{Cognome} di \textbf{Artista}
    
    \item \textit{Film} di \textbf{Recitazione} in \textit{ID} di \textbf{Film}
    
    \item \textit{NomeAttore}, \textit{CognomeAttore} di \textbf{Recitazione} in \textit{Nome}, \textit{Cognome} di \textbf{Artista}
    
    \item \textit{Critico} di \textbf{Critica} in \textit{Codice} di \textbf{Critico}
    
    \item \textit{Film} di \textbf{Critica} in \textit{ID} di \textbf{Film}
    
    \item \textit{Utente} di \textbf{Recensione} in \textit{Codice} di \textbf{Utente}
    
    \item \textit{Film} di \textbf{Recensione} in \textit{ID} di \textbf{Film}
    
    \item \textit{Genere} di \textbf{GenereFilm} in \textit{Nome} di \textbf{Genere}
    
    \item \textit{Film} di \textbf{GenereFilm} in \textit{ID} di \textbf{Film}
    
    \item \textit{Paese} di \textbf{CasaProduzione} in \textit{Codice} di \textbf{Paese}
    
    \item \textit{File} di \textbf{Doppiaggio} in \textit{ID} di \textbf{File}
    
    \item \textit{Lingua} di \textbf{Doppiaggio} in \textit{Nome} di \textbf{Lingua}
    
    \item \textit{File} di \textbf{Sottotitolaggio} in \textit{ID} di \textbf{File}
    
    \item \textit{Lingua} di \textbf{Sottotitolaggio} in \textit{Nome} di \textbf{Lingua}
    
    \item \textit{FamigliaAudio}, \textit{VersioneAudio} di \textbf{File} in \textit{Famiglia}, \textit{Versione} di 
    \textbf{\textit{FormatoCodifica}}
    
    \item \textit{FamigliaVideo}, \textit{VersioneVideo} di \textbf{File} in \textit{Famiglia}, \textit{Versione} di \textbf{FormatoCodifica}
    \item \textit{Edizione} di \textbf{File} in \textit{ID} di \textbf{Edizione}
    
    \item \textit{Film} di \textbf{Edizione} in \textit{ID} di \textbf{Film}
    
    \item \textit{Edizione} di \textbf{Restrizione} in \textit{ID} di \textbf{Edizione}
    
    \item \textit{Paese} di \textbf{Restrizione} in \textit{Codice} di \textbf{Paese}
    
    \item \textit{Paese} di \textbf{IP Range} in \textit{Codice} di \textbf{Paese}
    
    \item \textit{File} di \textbf{P.o.P} in \textit{ID} di \textbf{File}
    \item \textit{Server} di \textbf{P.o.P} in \textit{ID} di \textbf{Server}
    
    \item \textit{Paese} di \textbf{DistanzaPrecalcolata} in \textit{Codice} di \textbf{Paese}
    
    \item \textit{Server} di \textbf{DistanzaPrecalcolata} in \textit{Server} di \textbf{Paese}
    
    \item \textit{Edizione} di \textbf{Erogazione} in \textit{ID} di \textbf{Edizione}
    
    \item \textit{Timestamp}, \textit{Utente}, \textit{IP}, \textit{Edizione}, \textit{InizioConnessione} di \textbf{Erogazione} in \textit{Timestamp}, \textit{Utente}, \textit{IP}, \textit{Edizione}, \textit{InizioConnessione} di \textbf{Visualizzazione}
    
    \item \textit{Server} di \textbf{Erogazione} in \textit{ID} di \textbf{Server}
    
    \item \textit{Edizione} di \textbf{Visualizzazione} in \textit{ID} di \textbf{Edizione}
    
    \item \textit{Utente}, \textit{IP}, \textit{InizioConnessione} di \textbf{Visualizzazione} in \textit{Utente}, \textit{IP}, \textit{Inizio} di \textbf{Connessione}
    
    \item \textit{Utente} di \textbf{Connessione} in \textit{Codice} di \textbf{Utente}
    
    \item \textit{Abbonamento} di \textbf{Utente} in \textit{Tipo} di \textbf{Abbonamento}
    
    \item \textit{Abbonamento} di \textbf{Esclusione} in \textit{Tipo} di \textbf{Abbonamento}
    
    \item \textit{Genere} di \textbf{Esclusione} in \textit{Nome} di \textbf{Genere}
    
    \item \textit{Utente} di \textbf{Fattura} in \textit{Codice} di \textbf{Utente}
    
    \item \textit{CartaDiCredito} di \textbf{Fattura} in \textit{PAN} di \textbf{Carta di Credito}

    \item \textit{Film} di \textbf{Visualizzazioni Giornaliere} in \textit{ID} di \textbf{Film}
    
    \item \textit{Paese} di \textbf{Visualizzazioni Giornaliere} in \textit{Codice} di \textbf{Paese}
\end{enumerate}
\subsubsection{Vincoli di Tupla}
Ogni attributo che fa parte di una chiave ha il vincolo \textit{NOT NULL}. \\ 
% Fa eccezione solo \textit{CartaDiCredito} di \textbf{Fattura}, che avrà valore \textit{NULL} finché un pagamento non viene effettuato. \\ \\
% CartaDiCredito non e' identificatore esterno ma solo Foreign Key
I seguenti attributi hanno la possibilità di essere \textit{NULL}: \\
\begin{enumerate}
\itemsep-0.25em
    \item \textit{DataFine} di \textbf{IPRange}, quando un range è in corso di validità
    
    \item \textit{MaxBitRate} di \textbf{FormatoCodifica}, quando la metodologia in questione non pone massimi al \textit{BitRate} di un \textbf{File}
\end{enumerate}
I seguenti attributi, invece, non potranno essere NULL (anche se un'altra interpretazione potrebbe dargli questa possibilità) in quanto: \\
\begin{enumerate}
\itemsep-0.25em
    \item \textit{MaxOre} di \textbf{Abbonamento}, quando non vi è un limite sulle ore massime di streaming al giorno lo si imposta a 100
    
    \item \textit{GBMensili} di \textbf{Abbonamento}, quando non vi è un limite sulla mole massima di streaming al mese lo si imposta a 10 000
\end{enumerate}
Questa scelta è stata effettuata nell'ottica di semplificare i controlli di tali valori durante lo streaming. \\ \\
Sono individuati, oltretutto, i seguenti vincoli intra-relazionali: \\
\begin{enumerate}
\itemsep-0.25em
    \item L'istante di \textit{Fine} di una \textbf{Connessione} deve essere non minore di quello di \textit{Inizio}
    
    \item Se impostata, la \textit{DataFine} di un \textbf{IPRange} deve essere non minore di \textit{DataInizio}
    
    \item Un \textbf{IPRange} deve avere IP di \textit{Fine} non minore di quello di \textit{Inizio}
\end{enumerate}
\subsubsection{Vincoli di Dominio}
Sono individuati i seguenti vincoli sul dominio di alcuni attributi. \\ 
\begin{enumerate}
\itemsep-0.25em
    \item Una \textbf{Recensione} deve avere un \textit{Voto} compreso tra 0.0 e 5.0
    
    \item Una \textbf{Critica} deve avere un \textit{Voto} compreso tra 0.0 e 5.0
    
    \item In ogni momento, una \textbf{Critica} deve avere \textit{Data} non maggiore della data corrente
    
    \item In ogni momento il \textit{Timestamp} di una \textbf{Visualizzazione} deve essere non maggiore di \textit{CURRENT\_TIMESTAMP}
    
    \item L'istante di \textit{Fine} di una \textbf{Connessione} deve essere non minore di quello di \textit{Inizio}
    
    \item Sia l'istante di \textit{Inizio} che quello di \textit{Fine} di una \textbf{Connessione} devono essere, in ogni momento, non maggiori di \textit{CURRENT\_TIMESTAMP}
    
    \item La \textit{DataInizio} di \textbf{IPRange} deve essere non maggiore della data corrente
\end{enumerate}
\subsubsection{Vincoli Inter-Relazionali}
Sono individuati i seguenti vincoli Inter-Relazionali. \\
Il loro scopo è mantenere coerenza tra i vari dati memorizzati in relazioni diverse nella stessa Base di Dati. \\
\begin{enumerate}
\itemsep-0.25em
    \item Una \textbf{Critica} deve avere l'anno della      \textit{Data} non minore dell'\textit{Anno} di produzione del \textbf{Film} criticato
    
    \item Un'\textbf{Edizione} deve avere \textit{Anno} di pubblicazione non minore dell'\textit{Anno} di produzione del \textbf{Film}
    
    \item Un \textbf{File}, se ha associati \textbf{FormatoAudio} e \textbf{Videoentrambi} con un \textit{MaxBitRate} deve avere l'attributo \textit{BitRate} non maggiore della somma degli altri due
    
    \item L'anno del \textit{Timestamp} di una \textbf{Visualizzazione} deve essere non minore dell'\textit{Anno} di pubblicazione dell'\textbf{Edizione} visualizzata
    
    \item Il \textit{Timestamp} di una \textbf{Visualizzazione} deve essere compreso tra gli istanti di \textit{Inizio} e di \textit{Fine} della \textbf{Connessione} associata
    
    \item L'istante di \textit{Fine} di una \textbf{Connessione} deve essere non minore di quello di \textit{Inizio}
    
    \item Una \textbf{CartadiCredito} utilizzata in un \textbf{Pagamento} non deve essere scaduta nella \textbf{DataPagamento}
    
    \item Il timestamp \textit{InizioErogazione} di \textbf{Erogazione} non può essere maggiore di \textit{CURRENT\_TIMESTAMP} e non può essere minore del \textit{TimeStamp} di inizio della Visualizzazione associata
    
    \item In uno stesso istante non possono esistere duo o più \textbf{IPRange} validi che collidono. Due range collidono quando esiste un indirizzo IP che appartiene ad entrambi i range
\end{enumerate}
\subsection{Analisi delle Dipendenze Funzionali e Normalizzazione}
A questo punto si procede con l'analisi delle dipendenza funzionali delle relazioni ottenute dalla traduzione. Il risultato ottenuto, come di seguito, è in forma BCNF. Per ogni tabella in elenco, si analizzano le dipendenza funzionali in essa presenti e si inserisce un \includegraphics[width=1em]{assets/CheckMark.png} ove la BCNF viene soddisfatta, altrimenti si opera una decomposizione. \\ \\
\begin{itemize}

    \item[•] \textbf{Film}(\underline{ID}, Titolo, Descrizione, Anno, CasaProduzione, NomeRegista, CognomeRegista, MediaRecensioni, NumeroRecensioni) \includegraphics[width=1em]{assets/CheckMark.png} \\
    {\small \textit{\underline{ID} \textrightarrow\, Titolo, Descrizione, Anno, CasaProduzione, NomeRegista, \\ CognomeRegista, MediaRecensioni, NumeroRecensioni}} 

    \item[•] \textbf{Artista}(\underline{Nome}, \underline{Cognome}, Popolarità) \includegraphics[width=1em]{assets/CheckMark.png} \\ 
    {\small \textit{\underline{Nome}, \underline{Cognome} \textrightarrow\, Popolarità}} 
    
    \item[•] \textbf{VincitaPremio}(\underline{Macrotipo}, \underline{Microtipo}, \underline{Data}, Film, NomeArtista, CognomeArtista) \includegraphics[width=1em]{assets/CheckMark.png} \\
    {\small \textit{\underline{Macrotipo}, \underline{Microtipo}, \underline{Data} \textrightarrow\, Film, NomeArtista, CognomeArtista}} 
    
    \item[•] \textbf{Recitazione}(\underline{Film}, \underline{NomeAttore}, \underline{CognomeAttore}) \includegraphics[width=1em]{assets/CheckMark.png}
    
    \item[•] \textbf{Critico}(\underline{Codice}, Nome, Cognome) \includegraphics[width=1em]{assets/CheckMark.png} \\ 
    {\small \textit{\underline{Codice} \textrightarrow\, Nome, Cognome}} 
    
    \item[•] \textbf{Critica}(\underline{Critico}, \underline{Film}, Testo, Data, Voto) \includegraphics[width=1em]{assets/CheckMark.png} \\ 
    {\small \textit{\underline{Critico}, \underline{Film} \textrightarrow\, Testo, Data, Voto}} 
    
    \item[•] \textbf{Recensione}(\underline{Utente}, \underline{Film}, Voto) \includegraphics[width=1em]{assets/CheckMark.png} \\ 
    {\small \textit{\underline{Utente}, \underline{Film} \textrightarrow\, Voto}} 
    
    \item[•] \textbf{Genere}(\underline{Nome}) \includegraphics[width=1em]{assets/CheckMark.png}
    
    \item[•] \textbf{GenereFilm}(\underline{Genere}, \underline{Film}) \includegraphics[width=1em]{assets/CheckMark.png}
    
    \item[•] \textbf{CasaProduzione}(\underline{Nome}, Paese) \includegraphics[width=1em]{assets/CheckMark.png} \\
    {\small \textit{\underline{Nome} \textrightarrow\, Paese}} 

    \item[•]  \textbf{Lingua}(\underline{Nome}) \includegraphics[width=1em]{assets/CheckMark.png}

    \item[•] \textbf{Doppiaggio}(\underline{File}, \underline{Lingua}) \includegraphics[width=1em]{assets/CheckMark.png}

    \item[•] \textbf{Sottotitolaggio}(\underline{File}, \underline{Lingua}) \includegraphics[width=1em]{assets/CheckMark.png}

    \item[•] \textbf{Paese}(\underline{Codice}, Nome, Posizione) \includegraphics[width=1em]{assets/CheckMark.png} \\
    {\small \textit{\underline{Codice} \textrightarrow\, Nome, Posizione}} \\
    {\small \textit{\underline{Nome} \textrightarrow\, Codice}} 

    \item[•] \textbf{File}(\underline{ID}, Dimensione, BitRate, FormatoContenitore, FamigliaAudio, VersioneAudio, FamigliaVideo, VersioneVideo, Risoluzione, FPS, BitDepth, Frequenza, Edizione) \includegraphics[width=1em]{assets/CheckMark.png} \\ 
    {\small \textit{\underline{ID} \textrightarrow\, Dimensione, BitRate, FormatoContenitore, FamigliaAudio, VersioneAudio, FamigliaVideo, VersioneVideo, Risoluzione, FPS, BitDepth, Frequenza, Edizione}}

    \item[•] \textbf{FormatoCodifica}(\underline{Famiglia}, \underline{Versione}, Lossy, MaxBitrate) \includegraphics[width=1em]{assets/CheckMark.png} \\
    {\small \textit{\underline{Famiglia}, \underline{Versione} \textrightarrow\, Lossy, MaxBitrate}}

    \item[•] \textbf{Edizione}(\underline{ID}, Film, Anno, Tipo, Lunghezza, Rapporto D'Aspetto) \includegraphics[width=1em]{assets/CheckMark.png} \\
    {\small \textit{\underline{ID} \textrightarrow\, Film, Anno, Tipo, Lunghezza, Rapporto D'Aspetto)}}
    
    \item[•] \textbf{Restrizione}(\underline{Edizione}, \underline{Paese}) \includegraphics[width=1em]{assets/CheckMark.png} 
    
    \item[•] \textbf{IPRange}(\underline{Inizio}, \underline{Fine}, \underline{DataInizio}, DataFine, Paese) \includegraphics[width=1em]{assets/CheckMark.png} \\
    {\small \textit{\underline{Inizio}, \underline{Fine}, \underline{DataInizio} \textrightarrow\, DataFine, Paese)}}
    
    \item[•] \textbf{Server}(\underline{ID}, Risoluzione, LunghezzaBanda, MTU, MaxConnessioni, CaricoAttuale, Posizione) \includegraphics[width=1em]{assets/CheckMark.png} \\
    {\small \textit{\underline{ID} \textrightarrow\, LunghezzaBanda, MTU, MaxConnessioni, CaricoAttuale, Posizione)}}
    
    \item[•] \textbf{P.o.P.}(\underline{File}, \underline{Server}) \includegraphics[width=1em]{assets/CheckMark.png}
    
    \item[•] \textbf{DistanzaPrecalcolata}(\underline{Paese}, \underline{Server}, ValoreDistanza) \includegraphics[width=1em]{assets/CheckMark.png} \\
    {\small \textit{\underline{Paese}, \underline{Server} \textrightarrow\, ValoreDistanza}}

    \item[•] \textbf{Erogazione}(\underline{Timestamp}, \underline{Edizione}, \underline{Utente}, \underline{IP}, \underline{InizioConnessione}, InizioErogazione, Server) \includegraphics[width=1em]{assets/CheckMark.png} \\
    {\small \textit{\underline{Timestamp}, \underline{Edizione}, \underline{Utente}, \underline{IP}, \underline{InizioConnessione} \textrightarrow\, InizioErogazione, Server}}

    \item[•] \textbf{Visualizzazione}(\underline{Timestamp},\underline{Edizione},\underline{Utente},\underline{IP},\underline{InizioConnessione}) \includegraphics[width=1em]{assets/CheckMark.png}

    \item[•] \textbf{Connessione}(\underline{Utente}, \underline{IP}, \underline{Inizio}, Fine, Hardware) \includegraphics[width=1em]{assets/CheckMark.png} \\
    {\small \textit{\underline{Utente}, \underline{IP}, \underline{Inizio} \textrightarrow\, Fine, Hardware}} % \\ Da ricontrollare
    % Due Utenti diversi possono collegarsi dallo stesso IP nello stesso istante ?
    % {\small \textit{\underline{Utente}, \underline{IP}, \underline{Inizio} \textrightarrow\, Fine, Hardware}}

    \item[•] \textbf{Utente}(\underline{Codice}, Nome, Cognome, Email, Password, Abbonamento, DataInizioAbbonamento) \includegraphics[width=1em]{assets/CheckMark.png} \\
    {\small \textit{\underline{Codice} \textrightarrow\, Nome, Cognome, Email, Password, Abbonamento, DataInizioAbbonamento}}

    \item[•] \textbf{Abbonamento}(\underline{Tipo}, Tariffa, Durata, Definizione, Offline, MaxOre, GBMensili)\includegraphics[width=1em]{assets/CheckMark.png} \\
    {\small \textit{\underline{Tipo} \textrightarrow\, Tariffa, Durata, Definizione, Offline, MaxOre, GBMensili}}

    \item[•] \textbf{Esclusione}(\underline{Abbonamento}, \underline{Genere}) \includegraphics[width=1em]{assets/CheckMark.png} 

    \item[•] \textbf{Fattura}(\underline{ID}, Utente, DataEmissione, DataPagamento, CartaDiCredito) \includegraphics[width=1em]{assets/CheckMark.png} \\
    {\small \textit{\underline{ID} \textrightarrow\, Utente, DataEmissione, DataPagamento, CartaDiCredito}}

    \item[•] \textbf{CartaDiCredito}(\underline{PAN}, Scadenza, CVV) \includegraphics[width=1em]{assets/CheckMark.png} \\
    {\small \textit{\underline{PAN} \textrightarrow\, Scadenza, CVV}}

    \item[•] \textbf{Visualizzazioni Giornaliere}(\underline{Film}, \underline{Data}, \underline{Paese}, NumeroVisualizzazioni) \includegraphics[width=1em]{assets/CheckMark.png} \\
    {\small \textit{\underline{Film}, \underline{Data}, \underline{Paese} \textrightarrow\, NumeroVisualizzazioni}}
    
\end{itemize}
Per quanto riguarda la tabella \textit{Connessione} si potrebbe pensare che esista la dipendenza funzionale {\small \textit{IP, Inizio \textrightarrow Fine, Hardware}}, tuttavia qualora qualcuno si collegasse dallo stesso PC, aprendo due tab in due browser diversi nello stesso istante, avrei stesso IP e Inizio ma due User-Agent diversi, dunque due Hardware diversi. \\ 
\section{Raccomandazione Utente}
\subsection{Rating}
Il rating di un film è una valutazione espressa in numero di stelle, con la possibilità di averne mezze, che parte da un minimo di zero ed arriva ad un massimo di 5. \\
Ogni film possiede due diverse tipologie di rating, ossia il rating del film ed il rating personalizzato, la prima è una metrica oggettiva che è, dunque, fissa per ogni film, la seconda, invece, varia sulla base del singolo utente in quanto dipendente dalla "history" di quest'ultimo. \\ \\
Precedentemente si era già parlato di una forma di valutazione, espressa come media delle varie recensioni. Tuttavia, essa differisce dal rating non solo in merito al sistema di valutazione utilizzato, stelle invece che numerico, ma anche e sopratutto per le idee dietro di esse. Mentre la media valutazione è una misura molto superficiale, che corrisponde solamente al pensiero dei vari utenti, il rating si propone come una valutazione più sofisticata, complessa e sicuramente completa, infatti essa prende in considerazione una moltitudine di parametri, quali le recensioni di critici, le popolarità sia del regista che degli attori, il numero di premi vinti ed, eventualmente, le preferenze dell'utente.
\subsubsection{Rating del Film}
Il rating del film consente, come detto precedentemente, di dare una valutazione a tutto tondo e sopratutto oggettiva di un film. \\ \\ 
Siano $R_{U}$ la media delle recensione degli utenti, $R_{C}$ la media delle recensione dei critici, $RM_{U}$ la media delle recensione degli utenti più alta tra tutti i film aventi almeno un genere in comune con il film, $P_{A}$ la media delle popolarità degli attori, $P_{R}$ la media delle popolarità dei registi ed, infine, $PV$ i premi vinti dal film. \\ \\
L'algoritmo è il seguente: \\ 
\[ V = \frac{\floor*{0.5 * (R_{U} + R_{C}) + 0.1 * (P_{A} + P_{R}) + 0.1 * PV + (R_{U} / RM_{U})}}{2}\] \\
L'idea dietro di esso è assegnare un valore che va da 0 a 10 ai seguenti parametri: media recensioni e valutazioni, media popolarità attori e regista, numero di premi vinti e media recensioni rispetto al massimo per il genere. \\
Assegnati i valori occorre fare una media pesata coi seguenti pesi: 0.5, 0.2, 0.2 e 0.1, prendere la parte intera del risultato e dividerla per due (in questo modo otteniamo le mezze stelle per un giudizio che arriva al massimo a 5 stelle).
\subsubsection{Rating per Utente}
Il rating per utente consente, come detto precedentemente, di dare una valutazione soggettiva di un film, basandosi sulle preferenze dell'utente. \\ \\ 
Siano $G_{1}, G_{2}$ i 2 generi preferiti dell'utente, $A_{1}, A_{2}, A_{3}$ i 3 attori preferiti dell'utente, $L_{1}, L_{2}$ le 2 lingue, in cui l'audio è disponibile, preferite dell'utente, $R$ il regista preferito ed, infine, $RA$ il rapporto d'aspetto preferito. \\
Questi valori valgono 1 se sono presenti nel film altrimenti 0 (e.g. Se $G_{1}$ corrisponde a \textit{Fantascienza}, in ogni film di \textit{Fantascienza} $G_{1}$ vale 1) \\
Per preferiti si intende i più visti nell'ultimo mese. \\ \\ 
L'algoritmo è il seguente: \\
\[ V = \frac{\floor*{2 * G_{1} + G_{2} + 1.5 * A_{1} + A_{2} + 0.5 * A_{3} + L_{1} + L_{2} + R + RA}}{2}\] \\
L'idea dietro di esso è assegnare un bonus per ogni preferenza dell'utente che viene soddisfatta dal film in questione, ovviamente ad ogni preferenza viene attributo uno specifico peso in base alla rilevanza. La somma di tutti questi pesi equivale a 10, ossia il massimo valore ottenibile che corrisponde ad una valutazione di 5 stelle. \\
Sommati tutti i bonus occorre prendere la parte intera per poi dividerla per due (in questo modo otteniamo le mezze stelle per un giudizio che arriva al massimo a 5 stelle).
\subsubsection{Sistema di Raccomandazione}
\textbf{Descrizione}: Il sistema di raccomandazione di contenuti è in grado di, dato un utente e la sua "history" di visualizzazioni, consigliare e proporre film potenzialmente graditi da quest'ultimo. \\ 
Una metrica perfetta per misurare quanto uno specifico film possa essere potenzialmente apprezzato da un particolare utente è il suddetto meccanismo di \textbf{Rating Utente} in quanto riesce a coniugare alla perfezione i gusti soggettivi con l'esigenza di raccomandazione. \\ \\ 
\textbf{Input}: Codice dell'Utente e Numero di film richiesti \\
\textbf{Output}: Codici dei Film Raccomandati \\ \\ 
\begin{lstlisting}
DROP PROCEDURE IF EXISTS `RaccomandazioneContenuti`;
DELIMITER //
CREATE PROCEDURE `RaccomandazioneContenuti`(
    IN codice_utente VARCHAR(100),
    IN numero_film INT
)
BEGIN

    WITH
        FilmRatingUtente AS (
            SELECT
                ID,
                RatingUtente(ID, codice_utente) AS Rating
            FROM Film
        )
    SELECT ID
    FROM FilmRatingUtente
    ORDER BY Rating DESC, ID
    LIMIT numero_film;


END
//
DELIMITER ;
\end{lstlisting}
\textbf{Commento Codice}: 
Sfruttando la funzione RatingUtente è possibile creare una tabella \textit{T(Film, RatingUtente)}, sulla quale poi verrà operato un ordinamento decrescente in base al RatingUtente, usando ORDER BY, per poi andare a selezionare solo i primi N film graditi dall'utente, mediante LIMIT N. 

% 3 Generi Preferiti
% 3 Attori Preferiti
% 2 Lingue Audio Preferita
% Regista Preferito
% Rapporto d'Aspetto Preferito
\section{Area Analytics}
\subsection{Classifiche}
\textbf{Descrizione}: Questa funzionalità si occupare di stilare delle classifiche di \textbf{Film} e di \textbf{Edizioni} più gettonate sulla base sia del \textbf{Paese} che dell'\textbf{Abbonamento}. L'idea dietro questa funzionalità è di poter comporre una classifica degli $N$ \textbf{Film} oppure \textbf{Edizioni} più visti in uno specifico \textbf{Paese} solo da \textbf{Utenti} con uno specifico \textbf{Abbonamento}. \\ \\
\textbf{Input}: $N$, Codice del Paese, Tipo di Abbonamento, $P$ (Variabile che determina cosa classificare) \\
\textbf{Output}: Classifica \\ \\
\begin{lstlisting}
CREATE PROCEDURE IF NOT EXISTS `Classifica`(
    N INT,
    codice_paese CHAR(2),
    tipo_abbonamento VARCHAR(50),
    P INT -- 1 -> Film   2 -> Edizioni
)
BEGIN

    IF p = 1 THEN

        WITH
            FilmVisualizzazioni AS (
                SELECT
                    E.Film,
                    COUNT(*) AS Visualizzazioni
                FROM Visualizzazione V
                INNER JOIN Utente U
                    ON V.Utente = U.Codice
                INNER JOIN Edizione E
                    ON E.ID = V.Edizione
                INNER JOIN IPRange IP
                    ON IP.Inizio <= V.IP AND IP.Fine >= V.IP AND IP.DataInizio <= V.InizioConnessione AND (IP.DataFine IS NULL OR IP.DataFine >= V.InizioConnessione)
                WHERE U.Abbonamento = tipo_abbonamento
                AND IP.Paese = codice_paese
                GROUP BY E.Film
            )
        SELECT
            Film
        FROM FilmVisualizzazioni
        ORDER BY Visualizzazioni DESC
        LIMIT N;

    ELSEIF p = 2 THEN

        WITH
            FilmVisualizzazioni AS (
                SELECT
                    V.Edizione,
                    COUNT(*) AS Visualizzazioni
                FROM Visualizzazione V
                INNER JOIN Utente U
                    ON V.Utente = U.Codice
                INNER JOIN IPRange IP
                    ON IP.Inizio <= V.IP AND IP.Fine >= V.IP AND IP.DataInizio <= V.InizioConnessione AND (IP.DataFine IS NULL OR IP.DataFine >= V.InizioConnessione)
                WHERE U.Abbonamento = tipo_abbonamento
                AND IP.Paese = codice_paese
                GROUP BY V.Edizione
            )
        SELECT
            Edizione
        FROM FilmVisualizzazioni
        ORDER BY Visualizzazioni DESC
        LIMIT N;

    ELSE

        SIGNAL SQLSTATE '45000'
        SET MESSAGE_TEXT = 'Parametro P non Valido';

    END IF;

END
\end{lstlisting}
\textbf{Commento Codice}: La stored procedure prende in input i parametri sopra descritti ed, in base al valore di $P$ decide come agire, ovvero se classificare i \textbf{Film} oppure le \textbf{Edizioni}. \\ \\
Successivamente, crea una \textbf{CTE} \textit{VisualizzazioniFilm} contenente tutti i Film con il relativo numero di visualizzazioni che soddisfano sia i criteri riguardanti la collocazione geografica che quelli riguardanti il tipo di abbonamento. \\ 
Per controllare il primo occorre fare un JOIN con la tabella \textbf{IPRange} selezionando il range valido. \\
Invece, per controllare il secondo occorre fare un \textit{JOIN} tra la tabella \textbf{Visualizzazione} e quella \textbf{Utente} per determinare se il tipo di abbonamento sia quello desiderato. \\ \\ 
Si conclude ordinando in base a Visualizzazioni e restituendo gli N valori più grandi. \\ \\
Qualora si volessero invece le \textbf{Edizioni} non occorre un \textit{JOIN} ma si procede direttamente col raggruppamento, l'ordinamento e la restituzione dei migliori N valori.
\subsection{Custom Analytics}
La Custom Analytics ideata è quella di \textit{Migliore Attore e Migliore Regista Inaspettato} della settimana.\\
Ogni settimana, considerando artisti aventi una popolarità minore di una certa soglia, viene stilata una classifica degli artisti e registi più gettonati. \\ \\ 
Questa funzionalità serve a dare rilievo e lustro ad artisti debuttanti, emergenti o semplicemente meno noti al grande pubblico. Siamo convinti che questa opzione susciterebbe indubbiamente grande interesse tra gli utenti di \textit{FilmSphere} in quanto sarebbe un modo per aprire i propri orizzonti ad artisti minori e, pertanto, ampliare la propria cultura cinematografica anche, magari, sviluppando un giudizio critico verso attori o registi debuttanti. \\ \\ 
Innanzitutto, occorre definire la soglia massima di popolarità tra gli artisti considerati, che decicidiamo di fissare a $2.5$ in quanto siamo particolarmente interessati ad artisti poco rilevanti. \\ 
Per quanto concerne il punteggio da attribuire ad un attore o regista vengono presi in considerazione tutti i \textbf{Film} in cui quest'ultimo ha recitato oppure ha diretto e si somma al suo punteggio la valutazione media del \textbf{Film}, a cui vanno aggiunti eventuali $5$ punti qualora il \textbf{Film} in questione risultase vincitore di un qualsivoglia tipo di premio. Inoltre, per ogni premio eventualmente vinto da egli (artista) vengono attribuiti 100 punti. \\ \\ 
Siano $v_{i}$ la valutazione media dell'i-esimo \textbf{Film} in cui ha partecipato l'\textbf{Artista}, $p_{i}$ una variabile che assume come valore $1$ se l'i-esimo film ha vinto almeno un premio e $0$ altrimenti, $n$ il numero di premi vinti dall'\textbf{Artista}, allora il suo punteggio $P$ è uguale a:
\[ P = \sum_{i=1}^{n} (v_{i} + 5 * p_{i}) + 100 * n \]
\textbf{Implementazione}: 
\begin{lstlisting}
CREATE FUNCTION `ValutazioneRegista`(
    Nome VARCHAR(50),
    Cognome VARCHAR(50)
    )
RETURNS FLOAT DETERMINISTIC
BEGIN

    DECLARE sum_v FLOAT;
    DECLARE sum_p FLOAT;
    DECLARE n INT;

    SET sum_v := (
        SELECT
            SUM(MediaRecensioni)
        FROM Film
        WHERE NomeRegista = Nome AND CognomeRegista = Cognome
    );

    SET sum_p := (
        SELECT
            COUNT(DISTINCT VP.Film)
        FROM Film F
        INNER JOIN VincitaPremio VP
            ON VP.Film = F.ID
        WHERE F.NomeRegista = Nome AND F.CognomeRegista = Cognome
    );

    SET n := (
        SELECT
            COUNT(*)
        FROM VincitaPremio
        WHERE NomeArtista = Nome AND CognomeArtista = CognomeArtista
    );

    RETURN sum_v + sum_p * 5 + n * 100.0;


\end{lstlisting} 
Questa è la funzione che permette, nel caso di un regista, di calcolare il suo punteggio. Per ottenere la somma della media delle recensioni dei film in cui ha recitato si accede direttamente a film e la si legge, per il numero di film vincitori di premi in cui ha visitato, invece, si accede a film, imponendo che il regista del film sia il regista "target", e si fa un JOIN con \textbf{VincitaPremio} per selezionare solamente i film vincitori di premi. \\
Infine, per ottenere il numero di premi vinti dal regista si accede direttamente a \textbf{VincitaPremio}. \\ \\ \\ 
\begin{lstlisting}
DROP PROCEDURE IF EXISTS `MiglioreRegista`;
DELIMITER //
CREATE PROCEDURE IF NOT EXISTS `MiglioreRegista`()
BEGIN

    WITH
        RegistaValutazione AS (
            SELECT
                Nome, Cognome,
                ValutazioneRegista(Nome, Cognome) AS Valutazione
            FROM Artista
            WHERE Popolarita <= 2.5
        )
    SELECT
        Nome, Cognome
    FROM RegistaValutazione
    WHERE Valutazione = (
        SELECT MAX(Valutazione)
        FROM RegistaValutazione
    );

END
//
DELIMITER ;
\end{lstlisting}
Questa è la procedura che ritorna il migliore, o i migliori in caso di ex aequo, regista. Innanzitutto, si procede creando un CTE contente tutti i registi con una popolarità non sopra il 2.5, assieme alla loro valutazione ottenuta con la funzione precedente. \\
A questo punto, si selezionano solamente i registi avente Valutazione massima.
% Considerare di dare un bonus per i Film più visti
\subsection{Bilanciamento carico}
\textbf{Descrizione}: Per prevenire sovraccarichi sui \textbf{Server}, si cerca di prevedere le \textbf{Edizoni} che saranno più richieste nel futuro prossimo in un determinato Paese, in modo da copiare/spostare i \textbf{File} di tali \textbf{Edizioni} sui \textbf{Server} più vicini a quel \textbf{Paese}. \\ \\ 
\textbf{Input}: N, M numeri naturali\\
\textbf{Output}: Lista di spostamenti da effettuare \\ \\
La procedura funziona al seguente modo: \\
\begin{enumerate}
    \item Per ogni \textbf{Paese} si individuano le \textbf{Visualizzazioni} avvenute in quel \textbf{Paese} dall'\textit{IP}
    \item Si raggruppa in base al \textbf{Paese} e all'\textbf{Edizione} visualizzata contando le visualizzazioni per ogni \textbf{Edizione} in ogni \textbf{Paese}
    \item Per ogni \textbf{Paese} si selezionano le \textit{N} \textbf{Edizioni} più visualizzate
    \item Per ogni \textbf{Paese} si individuano gli \textit{M} \textbf{Server} più vicini
    \item Ogni coppia Edizione-Server, ignorando il \textbf{Paese} che non serve più, viene aggiunta, con attenzione a non aggiungere duplicati, alla lista da restituire
    \item Per ogni \textbf{Edizione} si individuano tutti i \textbf{File} e si trasforma la lista in una File-Server; si scartano le coppie dove esiste già un \textbf{P.o.P.} tra quel \textbf{File} e quel \textbf{Server}
\end{enumerate}
Allo stesso modo si può utilizzare lo stesso criterio per individuare i \textbf{File} che non è necessario tenere su un \textbf{Server}, seguendo la stessa procedura ma scegliendo, invece delle più visualizzate, le \textbf{Edizioni} meno visualizzate e, invece di prendere le coppie dove non vi è un \textbf{P.o.P.}, si prendono solo le coppie dove esiste già il \textbf{P.o.P.} \\
\section{Area Streaming} 
\subsection{Individuazione Server}
\textbf{Descrizione}: Quando un \textbf{Utente} richiede di guardare un'\textbf{Edizione}, l'applicazione invia la richiesta al DBMS che deve restituire il miglior Server al quale connettersi. \\ \\
\textbf{Input}: \textit{Codice} \textbf{Utente}, \textit{ID} dell'\textbf{Edizione}, \textit{IP} dell'Utente, MaxBitRate, MaxRis, ListaEncodingAudio, ListaEncodingVideo \\
\textbf{Output}: \textit{ID} del \textbf{Server} migliore assieme all'\textit{ID} del \textbf{File} migliore o nulla in caso di non disponibilità \\ \\
Dei parametri passati alla procedura, \textit{Codice}, \textit{ID}, \textit{IP} servono ad individuare l'\textbf{Utente}, il suo \textbf{Abbonamento} e le relative permissioni, il \textbf{Paese} dove si trova, mentre \textit{MaxBitRate} e \textit{MaxRis} sono parametri determinati dalle caratteristiche del dispositivo e dalla potenza di segnale attuale della rete al quale si è connessi. Tali parametri sono calcolati dall'applicazione lato client e servono per scegliere il \textbf{File} dell'\textbf{Edizione} più adatto. \\ 
ListaEncodingAudio e ListaEncodingVideo sono strinche contenenti le famiglie di \textbf{FormatoCodifica} che il dispositivo client supporta per mostrare il contenuto. La selezione si assicura che il \textbf{File} rispetti queste caratteristiche. \\ \\
La procedura seguita per l'individuazione del Server migliore è la seguente: \\ \\
\begin{enumerate}
    \item Vengono lette l'entità \textbf{Utente} e il relativo \textit{Abbonamento}, il \textbf{Film} associato all'\textbf{Edizione} e i \textbf{Generi} del \textbf{Film} e, se esistono, le \textbf{Esclusioni} tra l'\textbf{Abbonamento} e il \textbf{Genere}. Se l'interrogazione restituisce dei valori allora la procedura termina: non si è autorizzati a guardare il contenuto. \\
    \item Dall'\textit{IP} si risale al \textit{Codice} di \textbf{Paese} tramite gli \textbf{IPRange} e si controlla se vi siano \textbf{Restrizioni} tra il \textbf{Paese} e l'\textbf{Edizione}. Se vi sono restrizioni la procedura termina: non si è autorizzati (per legge) a guardare il contentuo. \\
    \item Si individuano tutti i \textbf{File} dell'\textbf{Edizione} che hanno codifiche supportate dal client
    \item Per ogni \textbf{File} trovato, si calcolano i seguenti parametri: \\  
    \[
    \Delta Rate = \begin{cases} 
        MaxBitRate - File.BitRate\text{ se }File.BitRate < MaxBitRate \\
        - 2\,(MaxBitRate - File.BitRate)\text{ altrimenti}
    \end{cases} 
    \]
    \[
    \Delta Ris = \begin{cases} 
        MaxRis - File.Risoluzione\text{ se }File.Risoluzione < MaxRis \\
        - 2\,(MaxRis - File.Risoluzione)\text{ altrimenti}
    \end{cases} 
    \] \\
    Tali valori serviranno poi nella scelta del \textbf{File} migliore, più alti sono questi valori, minori saranno le probabilità che il \textbf{File} sia scelto \\
    \item Per ogni \textbf{File} vengono individuati i \textbf{P.o.P.} e da essi i \textbf{Server} che li contengono. Dal \textbf{Paese} e i \textbf{Server}, passando per \textbf{DistanzaPrecalcolata} si ottiene la distanza, in km, tra la capitale del \textbf{Paese} dell'\textbf{Utente} e ogni \textbf{Server}. Di ogni \textbf{Server} si estrae il \textit{CaricoPercentuale} col seguente modo: \\
    \[
    CaricoPercentuale = CaricoAttuale/MaxConnessioni
    \] 
    \item Si effettuano le seguenti operazioni matematiche per riscalare i valori trovati: \\
    \[
    \Delta Rate'\,=\,map(\Delta Rate,\,0,\,MAX\_RATE,\,0,\,w_{Rate}) 
    \]
    \[
    \Delta Ris'\,=\,map(\Delta Ris,\,0,\,MAX\_RIS,\,0,\,w_{Ris}) 
    \]
    \[
    \Delta Pos'\,=\,map(ValoreDistanza,\,0,\,MAX\_DIST,\,0,\,w_{Pos}) 
    \]
    \[
    CaricoPercentuale'\,=\,CaricoPercentuale\,*\,w_{Carico} 
    \]\\
    Dove \textit{map} è una funzione che riscala un numero appartenente ad un intervallo portandolo in in nuovo intervallo. \\ 
    Essa è calcolata al seguente modo: \\
    \[
    \begin{split}
    map(x,\,IN_{Min},\,IN_{Max},\,OUT_{Min},\,OUT_{Max})\,=\,OUT_{Min}\,+\\
    \frac{x\,-\,IN_{Min}}{IN_{Max}\,-\,IN_{Min}}\,\cdot\,(OUT_{Max}\,-\,OUT_{Min})
    \end{split}
    \] \\
    I valori \textit{MAX\_RATE}, \textit{MAX\_RIS} e \textit{MAX\_DIST} rappresentano invece il massimo valore che le relative grandezze possono raggiungere. \\ \\
    I valori $w_{Rate}$, $w_{Ris}$, $w_{Pos}$ e $w_{Carico}$ sono i pesi che permetteranno di confrontare i tali paraemtri tra di loro, potendo cambiare facilmente l'importanza di uno rispetto all'altro. \\ \\
    Lo score Finale di ogni combinazione File-Server è asseganto come
    \[
    Score\,=\Delta Rate'\,+\Delta Ris'\,+\Delta Pos'\,+\,CaricoPercentuale' 
    \]
    \item La combinazione con lo \textit{Score} minore sarà restituita all'\textbf{Utente}. 
\end{enumerate} 
Un esempio di valori dei pesi può essere:
\begin{itemize}
    \item $w_{Rate}\,=\,5$ in modo da essere un parametro poco significativo, 17\% del totale
    \item $w_{Ris}\,=\,3$ in modo da essere il parametro meno importante, 10\% del totale
    \item $w_{Pos}\,=\,12$ in modo da essere il più importante (la distanza determina latency durante la fruizione), 40\% del totale
    \item $w_{Carico}\,=\,10$ In modo da tenere il carico bilanciato tra i diversi \textbf{Server}, 33\% del totale
\end{itemize}
\subsection{Ribilanciamento del carico}
Durante lo streaming di contenuti, gli \textbf{Utenti} possono cambiare \textbf{Server} dal quale stanno erogando, su richiesta del DBMS stesso, in modo da non sovraccaricare il sistema. \\
Ogni 10 minuti (intervalli detti tick) il DBMS calcolerà la media di \textit{CaricoPercentuale} dei \textbf{Server} e se tale media è sopra il 70\%. \\
Si individuano i (massimo 3) \textbf{Server} con \textit{CaricoPercentuale} più alto e sopra la media e, per ciascuno di essi: \\
\begin{enumerate}
    \item Si leggono le \textbf{Erogazioni} attive che riguardano tale \textbf{Server} e un \textbf{File} che è presente anche in almeno un altro \textbf{Server} che non sia nei 3 ineterssati e, se è presente, dove è presente. \\
    Le \textbf{Erogazioni} considerate non devono essere iniziate negli ultimi 3 tick e si deve prevedere (in base al \textit{TimeStamp} di \textbf{Visualizzazione} e la \textit{Durata} di \textbf{Edizione}) che terminino entro il prossimo.
    
    \item Per ciascuna di queste erogazioni risalgo al \textbf{Paese} dove si trova l'Utente dall'\textit{IP} e dal \textbf{Paese} al \textit{ValoreDistanza} tra l'\textbf{Utente} e il \textbf{Server} attuale
    
    \item Per ciascuna di queste erogazioni in corso individuo la migliore alternativa utilizzando lo stesso criterio di scelta iniziale dei \textbf{File} e \textbf{Server}. Si pone attenzione al fare sì, anche in questo caso, che un \textbf{Utente} non possa ricevere un \textbf{File} con qualità maggiore a quella massima del suo abbonamento. Oltretutto durante la ricerca del \textbf{File} si escludono il \textbf{Server} carichi (quelli dai quali si sta cercando di rimuovere \textbf{Erogazioni}). \\
    Per ciascuna Alternativa individuata memorizzo anche il suo punteggio
    
    \item Ordinando in base al punteggio, scelgo le migliori coppie Erogazione-Alternativa in numero minore al 5\% del \textit{MaxConnessioni} del \textbf{Server}: Per ogni Server scelgo le migliori N o meno Alternative, vove N è il \% del \textit{MaxConnessioni}. \\ \\
    Il risultato viene messo in una tabella (dal volume insignificante) che, tramite un'interrogazione avviata dai server (macchine) stessi, sarà svuotata e gli \textbf{Utenti} riassegnati, modificando solo dopo le \textbf{Erogazioni} anche nel DBMS
\end{enumerate}
\subsection{Caching previsionale}
\textbf{Descrizione}: Si ipotizza che ogni \textbf{File} sia presente in un hub centrale, macchina che ha il solo scopo di trasmettere tali contenuti ai \textbf{ Server}, creando dosì i \textbf{P.o.P.} \\
In modo da facilitare le operazioni agli algoritmi di \textit{Individuazione Server} e di \textit{Ribilanciamento di quest'ultimi} viene creato anche un \textit{Bilanciamento Preventivo}: algoritmo che cerca di prevedere i contenuti che avranno grande domanda nel prossimo futuro e, con questi dati, cerca di indicare i \textbf{Server} nei quali è più opportuno effettuare un \textbf{P.o.P.} di tali \textbf{File} creando così un sistema di Caching previsionale. \\ Allo stesso modo l'algoritmo individua i \textbf{File} che vranno minore probabilità di essere visualizzati in determinati \textbf{Server}, consigliando quindi di rimuoverli per fare spazio ad altri contenuti se lo spazio sulla macchina si dovesse esaurire. \\ \\
La nostra metrica misurante la probabilità che ha un certo File di essere visualizzato da un certo Utente è la seguente: \\ Considerati i primi 10 Film secondo la raccomandazione utente associamo ad ognuno di essi le seguenti probabilità decrescenti: 30\%, 22\%, 11\%, 9\%, 8\%, 6\%, 5\%, 4\%, 3\%, 2\% \\ 
Queste probabilità vanno poi ripartite tra i vari File associati al Film, considerando che si suddividano in maniera equiprobabile \\ \\
La procedura di individuazione di contenuti da aggiungere funziona al seguente modo: \\
\textbf{Input}: X, N, M numeri interi non grandi \\
\textbf{Output}: Lista di spostamenti consigliati \\
\begin{enumerate}
    \item Per ogni \textbf{Utente} si considera il \textbf{Paese} dal quale si connette di più e dal Paese gli N \textbf{Server} più vicini
    
    \item Per ogni coppia \textbf{Utente}, \textbf{Paese} si considerano gli M \textbf{File} con probabilità maggiore di essere guardati, ciascuno con la probabilità di essere guardato
    
    \item si raggruppa in base al \textbf{Server} e ad ogni \textbf{File}, sommando, per ogni \textbf{Server}-\textbf{File} la probabilità che sia guardato dall'\textbf{Utente} moltiplicata per un numero che scala in maniera decrescente in base al \textit{ValoreDistanza} tra Paese e Server

    \item Si restituiscono le prime X coppie Server-File con somma maggiore per le quali non esiste già un \textbf{P.o.P.}
\end{enumerate}
La procedura di individuazione di contenuti da rimuovere funziona allo stesso modo di quella di individuazione di contenuti da raggiungere ma all'ultimo punto vengono restituite le X coppie con somma più bassa dove è presente un \textbf{P.o.P.} \\

% - - - Raccomandazione Contenuti - - -
% Semplicemenete si fa una classifica in base al Rating per Utente
% - - - - - - - - - - - - - - - - - - -

% - - 


% - - - Bilanciamento del Carico - - -
% Per ogni paese spostare X file delle N edizioni piu' viste in quel Paese nei primi M server piu' vicini
% Per ogni server, tranne il "main", eliminare i file delle N edizioni meno visualizzate negli M paesi piu' vicini
% - - - - - - - - - - - - - - - - - -

% - - - - Caching - - - -
% Dato un Utente, consideriamo i primi N server piu' vicino ad esso ed i primi M Film raccomandati all'utente.
% Per ogni Coppia (Server, Film) si va ad incrementare un contatore che riguarda questo coppia
% Si reitera il processo per tutti gli utenti 
% Si prendono le prime X coppie (Server, Film) con valore del Contatore Maggiore
% Per ogni coppia si spostano alcune Edizioni di quel Film sul Server
% - - - - - - - - - - - - 

\section{Implementazione Fisica}
\subsection{Note su Vincoli}
Il vincolo riguardante il massimo bitarte per un file è stato implementato con un before trigger sia su inserimento che su aggiornamento. Le due ridondanze su \textbf{Film} sono mantenuto con vari trigger su inserimento, modifica e cancellazione su recensione. \\
\textbf{DistanzaPrecalcolata} viene mantenuta attraverso dei trigger sia su server che su paese. \textbf{CaricoAttuale} è stato implementato grazie ad un trigger di inserimento e cancellazione su erogazione. 
\subsection{Popolamento}
Il popolamento del database è avvenuto attraverso l'auisilio di svariati dataset reperibili online con l'aggiunta di brevi script in python per la generazione automatica di codice in sql. \\ \\
Per quanto riguarda gli \textbf{Utenti}, i codici e le email sono state selezionati da alcuni dataset esistenti \cite{nomi}, i nomi e cognomi sono stati generati prendendo un sottoinsieme del prodotto cartesiano tra un insieme di nomi e cognomi anch'essi reperiti online. I restanti campi sono stati generati casualmente attraverso gli script \cite{passmail}. \\ 
Tutta la sezione relativa a \textbf{Fattura}, \textbf{Pagamento}, \textbf{CartaDiCredito}, \textbf{Connessione} e \textbf{Visualizzazione} è stata generata in maniera aleatoria, sulla base del popolamento di utente, attraverso i suddetti script. \\ \\ 
La tabella \textbf{Film} è stata popolata sia attraverso dataset di film italiani che mediante l'assistenza di software di intelligenza artificiale, ossia ChatGPT. \\
La tabella \textbf{Vincita} è stata popolata sulla base di premi reali, mentre sia gli \textbf{Artisti} che i \textbf{Critici} sono stati generati grazie al sopracitato prodotto cartesiano, aggiungendo poi parametri ottenuti casualmente. \\
Le tabelle \textbf{Recitazione}, \textbf{Regia}, \textbf{Recensione} e \textbf{Critica} sono state popolate creando le varie corrispondenze in maniera aleatoria con degli script e includendo eventuali dati aggiuntivi o generati casualmente, come nel caso di Recensione, o presi da un ampio pool di possibilità, come nel caso di Critica. \\ \\
Sia \textbf{Paese} che \textbf{Lingua} sono stati popolati utilizzando alcuni dataset, mentre \textbf{CasaProduzione} è stato generato selezionando nomi esistenti ma creando associazioni in maniera aleatoria. \\
\textbf{Edizione} e \textbf{File} sono stati generati mediante l'utilizzo che ad ogni Film, o rispettivamente Edizione, associassero un numero variabile di Edizione o File, sempre in modo da rispettare i rapporti stimati in precedenza. I restanti parametri sono stati generati casualmente scegliendoli da un pool molto ampio di opzioni. \\ 
\textbf{FormatoCodifica} è stato generato a partire da u elenco di codec \cite{codecs}. \\
\textbf{Doppiaggio} e \textbf{Sottotitolaggio} sono stati popolati con associazioni casuali mediante l'utilizzo di brevi script.
\subsection{Trigger e Event}
Nel database sono stati implementati alcuni trigger ed event, di seguito una breve lista con una descrizione sommaria.
\subsubsection{AnnoEdizioneValido}
\textbf{Tabella}: Edizione \\
\textbf{Tipo}: Before Insert\\
\textbf{Descrizione}: Controlla la validità di anno di edizione \\
\subsubsection{DataCriticaValido}
\textbf{Tabella}: Critica \\
\textbf{Tipo}: Before Insert\\
\textbf{Descrizione}: Controlla la validità della data di critica \\
\subsubsection{InserimentoFile}
\textbf{Tabella}: File \\
\textbf{Tipo}: Before Insert\\
\textbf{Descrizione}: Controlla la validità di bitrate \\
\subsubsection{ModificaFile}
\textbf{Tabella}: File \\
\textbf{Tipo}: Before Update\\
\textbf{Descrizione}: Controlla la validità di bitrate \\
\subsubsection{ModificaErogazione}
\textbf{Tabella}: Erogazione\\
\textbf{Tipo}: Before Update\\
\textbf{Descrizione}: Aggiorna il timestamp e gestisce i server\\
\subsubsection{AggiungiErogazione}
\textbf{Tabella}: Erogazione \\
\textbf{Tipo}: After Insert \\
\textbf{Descrizione}: Gestisce i server \\
\subsubsection{RimuoviErogazione}
\textbf{Tabella}: Erogazione \\
\textbf{Tipo}: After Delete \\
\textbf{Descrizione}: Gestisce i server \\
\subsubsection{IpRangeControlloAggiornamento}
\textbf{Tabella}: IPRange\\
\textbf{Tipo}: Before Update\\
\textbf{Descrizione}: Impedisce modifiche anomale agli IPRange \\
\subsubsection{InserimentoRecensione}
\textbf{Tabella}: Recensione \\
\textbf{Tipo}: After Insert \\
\textbf{Descrizione}: Gestisce la Ridondanza in Film \\
\subsubsection{ModificaRecensione}
\textbf{Tabella}: Recensione \\
\textbf{Tipo}: After Update \\
\textbf{Descrizione}: Gestisce la Ridondanza in Film \\
\subsubsection{CancellazioneRecensione}
\textbf{Tabella}: Recensione \\
\textbf{Tipo}: After Delete \\
\textbf{Descrizione}: Gestisce la Ridondanza in Film \\
\subsubsection{GestioneVisualizzazione}
\textbf{Frequenza}: Ogni Giorno \\
\textbf{Descrizione}: Elimina le Visualizzazioni scadute \\
\subsubsection{GestioneConnessione}
\textbf{Frequenza}: Ogni Giorno \\
\textbf{Descrizione}: Elimina le Connessioni scadute \\
\subsubsection{GestioneErogazione}
\textbf{Frequenza}: Ogni Ora \\
\textbf{Descrizione}: Elimina le Erogazioni terminate \\
\subsection{API}
Il database così strutturato oltre a fornire l'accesso ai raw data tramite
interrogazioni formulate in SQL espone le seguenti API sotto forma di stored
procedure o stored function:
% Stored Procedure
\subsubsection{AggiungiErogazioneServer}
\textbf{Descrizione}: Incrementa il carico attuale di un server di un'unità.\\
\textbf{Input}: ID del Server\\
\textbf{Output}: \\
\subsubsection{BilanciamentoDelCarico}
\textbf{Descrizione}: Prevede le Edizioni che potrebbero essere richieste di frequente da un Paese, fornendo una lista di File e Server contenente gli spostamenti consigliati.\\
\textbf{Input}: N, M numeri naturali\\
\textbf{Output}: Lista di Spostamenti \\
\subsubsection{CachingPrevisionale}
\textbf{Descrizione}: Prevede i File che potrebbero essere richiesti di frequente da un Utente, fornendo una lista di File e Server contenente gli spostamenti consigliati.\\
\textbf{Input}: X, N, M numeri naturali\\
\textbf{Output}: Lista di Spostamenti \\
\subsubsection{CalcolaDistanzaPaese}
\textbf{Descrizione}: Calcola le distanze tra un \textbf{Paese} e tutti i \textbf{Server}. Gli output vengono calcolati tramite ST\_DISTANCE\_SPHERE e divisi per 1000 e memorizzati nella tabella DistanzaPrecalcolata. La distanza dal \textbf{Paese} fittizio 'World' è 0 verso ogni \textbf{Server}. \\
\textbf{Input}: \textit{Codice} di \textbf{Paese}\\
\textbf{Output}: \\
\subsubsection{CalcolaDistanzaServer}
\textbf{Descrizione}: Calcola le distanze tra un \textbf{Server} e tutti i \textbf{Paesi}. Gli output vengono calcolati tramite ST\_DISTANCE\_SPHERE e divisi per 1000 e memorizzati nella tabella \textbf{DistanzaPrecalcolata}. Ogni Server ha distanza 0 verso il fittizio 'World'.\\
\textbf{Input}: \textit{ID} di \textbf{Server}\\
\textbf{Output}: \\
\subsubsection{CambioAbbonamento}
\textbf{Descrizione}: Cambia l'abbonamento di un utente dopo aver controllato che sia in pari coi pagamenti\\
\textbf{Input}: Codice di Utente, Tipo di Abbonamento \\ 
\textbf{Output}: \\
\subsubsection{FileMiglioreQualita}
\textbf{Descrizione}: Restituisce i File, di uno specifico film, con le risoluzioni piuuu alte che un utente puooo vedere.\\
\textbf{Input}: Codice di Utente, ID di Film\\
\textbf{Output}: ID di Film, Risoluzione \\
\subsubsection{FilmDisponibiliInLinguaSpecifica}
\textbf{Descrizione}: Calcola tutti i Film disponibili in una Lingua.\\
\textbf{Input}: Lingua \\
\textbf{Output}: Lista di Film \\
\subsubsection{FilmEsclusiAbbonamento}
\textbf{Descrizione}: Calcola il numero di Film esclusi da un abbonamento.\\
\textbf{Input}: Abbonamento \\
\textbf{Output}: Numero di Film esclusi \\
\subsubsection{FilmPiuVistiRecentemente}
\textbf{Descrizione}: Calcola i Film piuuu visti di recente.\\
\textbf{Input}: Numero Massimo di Film \\
\textbf{Output}: Lista di Film \\
\subsubsection{GeneriDiUnFilm}
\textbf{Descrizione}: Calcola i Generi a cui appartiene un Film e determina se l'utente è abilitato o meno alla visione di quel film.\\
\textbf{Input}: Film, Abbonamento \\
\textbf{Output}: Lista di Generi, Abilitazione \\
\subsubsection{RimuoviErogazioneServer}
\textbf{Descrizione}: Decrementa il carico attuale di un server di un'unità.\\
\textbf{Input}: ID del Server\\
\textbf{Output}: \\
\subsubsection{VinciteDiUnFilm}
\textbf{Descrizione}: Calcola tutti i premi vinti da un Film.\\
\textbf{Input}: Film \\
\textbf{Output}: Lista di Premi, Numero di Premi \\
% Stored Function
\subsubsection{Ip2Paese}
\textbf{Descrizione}: Chiama \textit{Ip2PaeseStorico} passandogli come parametri l'\textit{IP} ricevuto e \textit{CURRENT\_TIMESTAMP} \\
\textbf{Input}: L'\textit{IP} di cui trovare il paese di appartenenza \\
\textbf{Output}: Il \textit{Codice} del \textbf{Paese} o '??' in caso di nessun riscontro\\
\subsubsection{Ip2PaeseStorico}
\textbf{Descrizione}: Dato l'\textit{IP} da cercare e l'istante temporale nel quale lo si desidera cercare cerca tra gli \textbf{IPRange} uno che soddisfi le condizioni. \\
\textbf{Input}: L'\textit{IP} di cui trovare il paese di appartenenza \\
\textbf{Output}: Il \textit{Codice} del \textbf{Paese} o '??' in caso di nessun riscontro\\
\subsubsection{IpAppartieneRangeInData}
\textbf{Descrizione}: Dati tutti gli estremi (sia numerici che temporali) di un IPRange, assieme ad un IP ed un istante temporale, controlla che l'IP appartenga al range dato nell'istante dato\\
\textbf{Input}: Inizio, Fine, DataInizio, DataFine, IP, DataDaControllare \\
\textbf{Output}: TRUE se l'IP appartiene a [Inizio, Fine] e il range è valido nell'istante richiesto, FALSE altrimenti\\
\subsubsection{IpRangeCollidono}
\textbf{Descrizione}: Controlla se esiste almeno un indirizzo che appartiene ad entrambi i range\\
\textbf{Input}: \textit{Inizio1}, \textit{Fine1}, \textit{Inizio2}, \textit{Fine2}. Per entrambi i range si assume \textit{Fine} > \textit{Inizio} per ipotesi \\
\textbf{Output}: TRUE in caso di collisione, FALSE altrimenti \\
\subsubsection{IpRangeValidoInData}
\textbf{Descrizione}: Controlla se un determinato range di ip è valido in una determinata data. Passare NULL al secondo o terzo parametro viene interpretato come CURRENT\_TIMESTAMP\\
\textbf{Input}: InizioValidità, FineValidità, IstanteDaControllare \\
\textbf{Output}: TRUE se FineValidità $ \geq $ IstanteDaControllare $ \geq $  InizioValidità, FALSE altrimenti \\
\subsubsection{IpRangePossoInserire}
\textbf{Descrizione}: Controlla se un range è già prensete nel DB\\
\textbf{Input}: \textit{NewInizio}, \textit{NewFine}, \textit{NewDataInizio}, \textit{NewPaese} \\
\textbf{Output}: TRUE se nell'istante NewDataInizio non vi è nessun Range con estremi e Paese uguali a quelli inseriti \\
\subsubsection{IpRangeInserisciFidato}
\textbf{Descrizione}: Inserisce un \textbf{IPRange} nel DB. Si può specificare se invalidare gli altri ip che collidno con ilnuovo inseirto o se non effettuare proprio tale controllo. \\
\textbf{Input}: \textit{NewInizio}, \textit{NewFine}, \textit{NewDataInizio}, \textit{NewDataFine}, \textit{NewPaese}, \textit{InvalidaCollisioni}  \\
\textbf{Output}: \\
\subsubsection{IpRangeInserisciAdessoFidato}
\textbf{Descrizione}: Inserisce un \textbf{IPRange} nel DB. Si può specificare se invalidare gli altri ip che collidno con ilnuovo inseirto o se non effettuare proprio tale controllo. \textit{DataInizio} è considerata CURRENT\_TIMESTAMP e \textit{DataFine} NULL\\
\textbf{Input}: \textit{NewInizio}, \textit{NewFine}, \textit{NewPaese}, \textit{InvalidaCollisioni}  \\
\textbf{Output}: \\
\subsubsection{IpRangeProvaInserire}
\textbf{Descrizione}: Inserisce un \textbf{IPRange} nel DB solo se non è già presente uno uguale o equivalente. \\
\textbf{Input}: \textit{NewInizio}, \textit{NewFine}, \textit{NewDataInizio}, \textit{NewDataFine}, \textit{NewPaese}  \\
\textbf{Output}: \\
\subsubsection{IpRangeProvaInserireAdesso}
\textbf{Descrizione}: Inserisce un \textbf{IPRange} nel DB solo se non è già presente uno uguale o equivalente. \textit{DataInizio} è considerata CURRENT\_TIMESTAMP e \textit{DataFine} NULL\\
\textbf{Input}: \textit{NewInizio}, \textit{NewFine}, \textit{NewPaese}  \\
\textbf{Output}: \\
\subsubsection{VisualizzazioniGiornaliereBuild}
\textbf{Descrizione}: Aggiunge i dati relativi alla giornata di ieri nell MV \textbf{VisualizzazioniGiornaliere}. Fa ciò solo se tali dati non sono già presenti. Viene chiamata con cadenza giornaliera dall'event \textbf{VisualizzazioniGiornaliereEvent} \\
\textbf{Input}: \\
\textbf{Output}: Solleva lo warning 'Procedura già lanciata oggi!' se i dati sono già presenti \\
\subsubsection{VisualizzazioniGiornaliereFullBuild}
\textbf{Descrizione}: Ricostruisce in toto la MV \textbf{VisualizzazioniGiornaliere}. Sovrascrive eventuali dati già presenti. Tale operazione può impiegare molto tempo: la procedura deve essere utilizzata solo dopo un'eventuale perdita dei dati della ridondanza. \\
\textbf{Input}: \\
\textbf{Output}: \\
\subsubsection{MathMap}
\textbf{Descrizione}: Porta un numero da un intervallo in un altro. \\
\textbf{Input}: X, inMin, inMax, outMin, outMax \\
\textbf{Output}: outMin + (outMax - outMin) * (x - inMin) / (inMax - inMin)\\
\subsubsection{StrListContains}
\textbf{Descrizione}: Separa la stringa in base alla virgola e controlla se almeno una delle sottostringhe ottenute contiene il termine richiesto. Per le sottostringhe gli spazi iniziali e finali vengono ignorati nella comparazione \\
\textbf{Input}: \textit{Pagliaio}, \textit{Ago} \\
\textbf{Output}: TRUE se \textit{Ago} è uno dei termini di \textit{Pagliaio}\\
\subsubsection{CalcolaDelta}
\textbf{Descrizione}: Dato un valore massimo \textit{Max} e un numero \textit{Valore} restituisce \textit{Max} - \textit{Valore} se \textit{Max} > \textit{Valore} \textit{}o 2(\textit{Valore} - \textit{Max}) negli altri casi. NULL a \textit{Max} viene considerato 0. \\
\textbf{Input}: \textit{Max}, \textit{Valore} \\
\textbf{Output}: Il Delta calcolato con il criterio sopra indicato \\
\subsubsection{MigliorServer}
\textbf{Descrizione}: Cerca di trovare la miglior coppia \textit{FileID} - \textit{ServerID} per un Utente che vorrebbe iniziare una \textbf{Visualizzazione} di un'\textbf{Edizione}. Viene controllato l'abbonamento dell'\textbf{Utente}, dalla quale si individua la massima qualità permessa. I \textbf{File} restituiti hanno codifiche Audio e Video supportate dal dispositivo del client. Il \textbf{Server} è selto in base alla prossimità e al suo carico attuale. La posizione dell'\textbf{Utente} è estratta dal suo indirizzo IP4. Nella richiesta, \textit{MaxBitRate} e \textit{MaxRisoluz} devono tenere conto dell'attuale stato della connessione tra il client e il server: sarebbe inutile trasmettere al client più dati di quanti ne può ricevere. \textit{ListaVideoEncodings} e \textit{ListaVideoEncodings} contengono le Famiglie dei FormatiCodifica supportati dal device client (che li concatena in automatico); NULL significa che ogni codifica è supportata. \\
\textbf{Input}: id\_utente, id\_edizione, ip\_connessione, MaxBitRate, MaxRisoluz, ListaVideoEncodings, ListaAudioEncodings\\
\textbf{Output}: FileID, ServerID \\
\subsubsection{TrovaMigliorServer}
\textbf{Descrizione}: Cerca di trovare il miglior trio \textit{FileID} - \textit{ServerID} - \textit{Punteggio} per un'\textbf{Edizione}. I \textbf{File} restituiti hanno codifiche Audio e Video supportate dal dispositivo del client. Il \textbf{Server} è selto in base alla prossimità e al suo carico attuale. La posizione dell'\textbf{Utente} è estratta dal suo indirizzo IP4. Nella richiesta, \textit{MaxBitRate} e \textit{MaxRisoluz} devono tenere conto dell'attuale stato della connessione tra il client e il server: sarebbe inutile trasmettere al client più dati di quanti ne può ricevere. \textit{ListaVideoEncodings} e \textit{ListaVideoEncodings} contengono le Famiglie dei FormatiCodifica supportati dal device client (che li concatena in automatico); NULL significa che ogni codifica è supportata. \\
\textbf{Input}: id\_edizione, paese\_utente, MaxRisoluzAbbonamento, MaxBitRate, MaxRisoluz, ListaVideoEncodings, ListaAudioEncodings\\
\textbf{Output}: FileID, ServerID, Punteggio \\
\subsubsection{RibilanciamentoCarico}
\textbf{Descrizione}: Individua i Server più stressati (sopra la media di carico) e, per ogni loro erogazione cerca di trovare alternative nei Server meno carichi. I risultati vengono messi nella tabella \textbf{ModificaErogazioni}, in attesa che le erogazioni vengano modificate anche nella realtà (i server stessi se ne occuperanno d'ora in avanti). Vengono considerate solo le Erogazioni non spostate nell'ultima mezz'ora e che non termineranno a breve. \\
\textbf{Input}: \\
\textbf{Output}: Riferimento a Erogazione, Server (server in utilizzo), Alternativa (altro server), File (proposta di alternativa) \\
\subsubsection{RatingUtente}
\textbf{Descrizione}: Fornisce un Rating, in stelle, ad un film in maniera oggettiva \\
\textbf{Input}:  ID di Film \\
\textbf{Output}: Valutazione \\
\newpage
\printbibliography
\end{document}
